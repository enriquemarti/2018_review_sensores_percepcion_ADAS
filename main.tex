
%% bare_jrnl.tex
%% V1.4b
%% 2015/08/26
%% by Michael Shell
%% see http://www.michaelshell.org/
%% for current contact information.
%%
%% This is a skeleton file demonstrating the use of IEEEtran.cls
%% (requires IEEEtran.cls version 1.8b or later) with an IEEE
%% journal paper.
%%
%% Support sites:
%% http://www.michaelshell.org/tex/ieeetran/
%% http://www.ctan.org/pkg/ieeetran
%% and
%% http://www.ieee.org/ 

%%*************************************************************************
%% Legal Notice:
%% This code is offered as-is without any warranty either expressed or
%% implied; without even the implied warranty of MERCHANTABILITY or
%% FITNESS FOR A PARTICULAR PURPOSE! 
%% User assumes all risk.
%% In no event shall the IEEE or any contributor to this code be liable for
%% any damages or losses, including, but not limited to, incidental,
%% consequential, or any other damages, resulting from the use or misuse
%% of any information contained here.
%%
%% All comments are the opinions of their respective authors and are not
%% necessarily endorsed by the IEEE.
%%
%% This work is distributed under the LaTeX Project Public License (LPPL)
%% ( http://www.latex-project.org/ ) version 1.3, and may be freely used,
%% distributed and modified. A copy of the LPPL, version 1.3, is included
%% in the base LaTeX documentation of all distributions of LaTeX released
%% 2003/12/01 or later.
%% Retain all contribution notices and credits.
%% ** Modified files should be clearly indicated as such, including  **
%% ** renaming them and changing author support contact information. **
%%*************************************************************************


% *** Authors should verify (and, if needed, correct) their LaTeX system  ***
% *** with the testflow diagnostic prior to trusting their LaTeX platform ***
% *** with production work. The IEEE's font choices and paper sizes can   ***
% *** trigger bugs that do not appear when using other class files.       ***                          ***
% The testflow support page is at:
% http://www.michaelshell.org/tex/testflow/



\documentclass[journal]{IEEEtran}
%
% If IEEEtran.cls has not been installed into the LaTeX system files,
% manually specify the path to it like:
% \documentclass[journal]{../sty/IEEEtran}





% Some very useful LaTeX packages include:
% (uncomment the ones you want to load)


% *** MISC UTILITY PACKAGES ***
%
%\usepackage{ifpdf}
% Heiko Oberdiek's ifpdf.sty is very useful if you need conditional
% compilation based on whether the output is pdf or dvi.
% usage:
% \ifpdf
%   % pdf code
% \else
%   % dvi code
% \fi
% The latest version of ifpdf.sty can be obtained from:
% http://www.ctan.org/pkg/ifpdf
% Also, note that IEEEtran.cls V1.7 and later provides a builtin
% \ifCLASSINFOpdf conditional that works the same way.
% When switching from latex to pdflatex and vice-versa, the compiler may
% have to be run twice to clear warning/error messages.






% *** CITATION PACKAGES ***
%
%\usepackage{cite}
% cite.sty was written by Donald Arseneau
% V1.6 and later of IEEEtran pre-defines the format of the cite.sty package
% \cite{} output to follow that of the IEEE. Loading the cite package will
% result in citation numbers being automatically sorted and properly
% "compressed/ranged". e.g., [1], [9], [2], [7], [5], [6] without using
% cite.sty will become [1], [2], [5]--[7], [9] using cite.sty. cite.sty's
% \cite will automatically add leading space, if needed. Use cite.sty's
% noadjust option (cite.sty V3.8 and later) if you want to turn this off
% such as if a citation ever needs to be enclosed in parenthesis.
% cite.sty is already installed on most LaTeX systems. Be sure and use
% version 5.0 (2009-03-20) and later if using hyperref.sty.
% The latest version can be obtained at:
% http://www.ctan.org/pkg/cite
% The documentation is contained in the cite.sty file itself.






% *** GRAPHICS RELATED PACKAGES ***
%
\ifCLASSINFOpdf
  % \usepackage[pdftex]{graphicx}
  % declare the path(s) where your graphic files are
  % \graphicspath{{../pdf/}{../jpeg/}}
  % and their extensions so you won't have to specify these with
  % every instance of \includegraphics
  % \DeclareGraphicsExtensions{.pdf,.jpeg,.png}
\else
  % or other class option (dvipsone, dvipdf, if not using dvips). graphicx
  % will default to the driver specified in the system graphics.cfg if no
  % driver is specified.
  % \usepackage[dvips]{graphicx}
  % declare the path(s) where your graphic files are
  % \graphicspath{{../eps/}}
  % and their extensions so you won't have to specify these with
  % every instance of \includegraphics
  % \DeclareGraphicsExtensions{.eps}
\fi
% graphicx was written by David Carlisle and Sebastian Rahtz. It is
% required if you want graphics, photos, etc. graphicx.sty is already
% installed on most LaTeX systems. The latest version and documentation
% can be obtained at: 
% http://www.ctan.org/pkg/graphicx
% Another good source of documentation is "Using Imported Graphics in
% LaTeX2e" by Keith Reckdahl which can be found at:
% http://www.ctan.org/pkg/epslatex
%
% latex, and pdflatex in dvi mode, support graphics in encapsulated
% postscript (.eps) format. pdflatex in pdf mode supports graphics
% in .pdf, .jpeg, .png and .mps (metapost) formats. Users should ensure
% that all non-photo figures use a vector format (.eps, .pdf, .mps) and
% not a bitmapped formats (.jpeg, .png). The IEEE frowns on bitmapped formats
% which can result in "jaggedy"/blurry rendering of lines and letters as
% well as large increases in file sizes.
%
% You can find documentation about the pdfTeX application at:
% http://www.tug.org/applications/pdftex





% *** MATH PACKAGES ***
%
%\usepackage{amsmath}
% A popular package from the American Mathematical Society that provides
% many useful and powerful commands for dealing with mathematics.
%
% Note that the amsmath package sets \interdisplaylinepenalty to 10000
% thus preventing page breaks from occurring within multiline equations. Use:
%\interdisplaylinepenalty=2500
% after loading amsmath to restore such page breaks as IEEEtran.cls normally
% does. amsmath.sty is already installed on most LaTeX systems. The latest
% version and documentation can be obtained at:
% http://www.ctan.org/pkg/amsmath




% *** SPECIALIZED LIST PACKAGES ***
%
%\usepackage{algorithmic}
% algorithmic.sty was written by Peter Williams and Rogerio Brito.
% This package provides an algorithmic environment fo describing algorithms.
% You can use the algorithmic environment in-text or within a figure
% environment to provide for a floating algorithm. Do NOT use the algorithm
% floating environment provided by algorithm.sty (by the same authors) or
% algorithm2e.sty (by Christophe Fiorio) as the IEEE does not use dedicated
% algorithm float types and packages that provide these will not provide
% correct IEEE style captions. The latest version and documentation of
% algorithmic.sty can be obtained at:
% http://www.ctan.org/pkg/algorithms
% Also of interest may be the (relatively newer and more customizable)
% algorithmicx.sty package by Szasz Janos:
% http://www.ctan.org/pkg/algorithmicx




% *** ALIGNMENT PACKAGES ***
%
%\usepackage{array}
% Frank Mittelbach's and David Carlisle's array.sty patches and improves
% the standard LaTeX2e array and tabular environments to provide better
% appearance and additional user controls. As the default LaTeX2e table
% generation code is lacking to the point of almost being broken with
% respect to the quality of the end results, all users are strongly
% advised to use an enhanced (at the very least that provided by array.sty)
% set of table tools. array.sty is already installed on most systems. The
% latest version and documentation can be obtained at:
% http://www.ctan.org/pkg/array


% IEEEtran contains the IEEEeqnarray family of commands that can be used to
% generate multiline equations as well as matrices, tables, etc., of high
% quality.




% *** SUBFIGURE PACKAGES ***
%\ifCLASSOPTIONcompsoc
%  \usepackage[caption=false,font=normalsize,labelfont=sf,textfont=sf]{subfig}
%\else
%  \usepackage[caption=false,font=footnotesize]{subfig}
%\fi
% subfig.sty, written by Steven Douglas Cochran, is the modern replacement
% for subfigure.sty, the latter of which is no longer maintained and is
% incompatible with some LaTeX packages including fixltx2e. However,
% subfig.sty requires and automatically loads Axel Sommerfeldt's caption.sty
% which will override IEEEtran.cls' handling of captions and this will result
% in non-IEEE style figure/table captions. To prevent this problem, be sure
% and invoke subfig.sty's "caption=false" package option (available since
% subfig.sty version 1.3, 2005/06/28) as this is will preserve IEEEtran.cls
% handling of captions.
% Note that the Computer Society format requires a larger sans serif font
% than the serif footnote size font used in traditional IEEE formatting
% and thus the need to invoke different subfig.sty package options depending
% on whether compsoc mode has been enabled.
%
% The latest version and documentation of subfig.sty can be obtained at:
% http://www.ctan.org/pkg/subfig




% *** FLOAT PACKAGES ***
%
%\usepackage{fixltx2e}
% fixltx2e, the successor to the earlier fix2col.sty, was written by
% Frank Mittelbach and David Carlisle. This package corrects a few problems
% in the LaTeX2e kernel, the most notable of which is that in current
% LaTeX2e releases, the ordering of single and double column floats is not
% guaranteed to be preserved. Thus, an unpatched LaTeX2e can allow a
% single column figure to be placed prior to an earlier double column
% figure.
% Be aware that LaTeX2e kernels dated 2015 and later have fixltx2e.sty's
% corrections already built into the system in which case a warning will
% be issued if an attempt is made to load fixltx2e.sty as it is no longer
% needed.
% The latest version and documentation can be found at:
% http://www.ctan.org/pkg/fixltx2e


%\usepackage{stfloats}
% stfloats.sty was written by Sigitas Tolusis. This package gives LaTeX2e
% the ability to do double column floats at the bottom of the page as well
% as the top. (e.g., "\begin{figure*}[!b]" is not normally possible in
% LaTeX2e). It also provides a command:
%\fnbelowfloat
% to enable the placement of footnotes below bottom floats (the standard
% LaTeX2e kernel puts them above bottom floats). This is an invasive package
% which rewrites many portions of the LaTeX2e float routines. It may not work
% with other packages that modify the LaTeX2e float routines. The latest
% version and documentation can be obtained at:
% http://www.ctan.org/pkg/stfloats
% Do not use the stfloats baselinefloat ability as the IEEE does not allow
% \baselineskip to stretch. Authors submitting work to the IEEE should note
% that the IEEE rarely uses double column equations and that authors should try
% to avoid such use. Do not be tempted to use the cuted.sty or midfloat.sty
% packages (also by Sigitas Tolusis) as the IEEE does not format its papers in
% such ways.
% Do not attempt to use stfloats with fixltx2e as they are incompatible.
% Instead, use Morten Hogholm'a dblfloatfix which combines the features
% of both fixltx2e and stfloats:
%
% \usepackage{dblfloatfix}
% The latest version can be found at:
% http://www.ctan.org/pkg/dblfloatfix

\usepackage{url}
\def\UrlBreaks{\do\/\do-}
\usepackage{breakurl}
\usepackage[draft,breaklinks]{hyperref}
%\usepackage{hyperref}
\usepackage{adjustbox}
\usepackage{multirow, array}
\usepackage{booktabs}
\newcommand{\ra}[1]{\renewcommand{\arraystretch}{#1}}
\usepackage{tabularx}
\usepackage{makecell}
\usepackage{float}
\newcolumntype{L}{>{\raggedright\arraybackslash}X}
\usepackage[acronym]{glossaries}
\usepackage{flushend}


%\ifCLASSOPTIONcaptionsoff
%  \usepackage[nomarkers]{endfloat}
% \let\MYoriglatexcaption\caption
% \renewcommand{\caption}[2][\relax]{\MYoriglatexcaption[#2]{#2}}
%\fi
% endfloat.sty was written by James Darrell McCauley, Jeff Goldberg and 
% Axel Sommerfeldt. This package may be useful when used in conjunction with 
% IEEEtran.cls'  captionsoff option. Some IEEE journals/societies require that
% submissions have lists of figures/tables at the end of the paper and that
% figures/tables without any captions are placed on a page by themselves at
% the end of the document. If needed, the draftcls IEEEtran class option or
% \CLASSINPUTbaselinestretch interface can be used to increase the line
% spacing as well. Be sure and use the nomarkers option of endfloat to
% prevent endfloat from "marking" where the figures would have been placed
% in the text. The two hack lines of code above are a slight modification of
% that suggested by in the endfloat docs (section 8.4.1) to ensure that
% the full captions always appear in the list of figures/tables - even if
% the user used the short optional argument of \caption[]{}.
% IEEE papers do not typically make use of \caption[]'s optional argument,
% so this should not be an issue. A similar trick can be used to disable
% captions of packages such as subfig.sty that lack options to turn off
% the subcaptions:
% For subfig.sty:
% \let\MYorigsubfloat\subfloat
% \renewcommand{\subfloat}[2][\relax]{\MYorigsubfloat[]{#2}}
% However, the above trick will not work if both optional arguments of
% the \subfloat command are used. Furthermore, there needs to be a
% description of each subfigure *somewhere* and endfloat does not add
% subfigure captions to its list of figures. Thus, the best approach is to
% avoid the use of subfigure captions (many IEEE journals avoid them anyway)
% and instead reference/explain all the subfigures within the main caption.
% The latest version of endfloat.sty and its documentation can obtained at:
% http://www.ctan.org/pkg/endfloat
%
% The IEEEtran \ifCLASSOPTIONcaptionsoff conditional can also be used
% later in the document, say, to conditionally put the References on a 
% page by themselves.




% *** PDF, URL AND HYPERLINK PACKAGES ***
%
%\usepackage{url}
% url.sty was written by Donald Arseneau. It provides better support for
% handling and breaking URLs. url.sty is already installed on most LaTeX
% systems. The latest version and documentation can be obtained at:
% http://www.ctan.org/pkg/url
% Basically, \url{my_url_here}.




% *** Do not adjust lengths that control margins, column widths, etc. ***
% *** Do not use packages that alter fonts (such as pslatex).         ***
% There should be no need to do such things with IEEEtran.cls V1.6 and later.
% (Unless specifically asked to do so by the journal or conference you plan
% to submit to, of course. )


% correct bad hyphenation here
\hyphenation{op-tical net-works semi-conduc-tor}


\begin{document}
%
% paper title
% Titles are generally capitalized except for words such as a, an, and, as,
% at, but, by, for, in, nor, of, on, or, the, to and up, which are usually
% not capitalized unless they are the first or last word of the title.
% Linebreaks \\ can be used within to get better formatting as desired.
% Do not put math or special symbols in the title.
\title{A Survey of Sensor Technologies for Perception in Automated Driving}
%
%
% author names and IEEE memberships
% note positions of commas and nonbreaking spaces ( ~ ) LaTeX will not break
% a structure at a ~ so this keeps an author's name from being broken across
% two lines.
% use \thanks{} to gain access to the first footnote area
% a separate \thanks must be used for each paragraph as LaTeX2e's \thanks
% was not built to handle multiple paragraphs
%

\author{Enrique~Mart\'i, %~\IEEEmembership{Member,~IEEE,}
        Miguel \'Angel~de Miguel, %~\IEEEmembership{Fellow,~OSA,}
		Fernando~Garc\'ia,~\IEEEmembership{Member,~IEEE,}
        and~Joshu\'e~P\'erez,%~\IEEEmembership{Life~Fellow,~IEEE}% <-this % 
        %%%stops a space
\thanks{E. Marti\'i and J. P\'erez work in the Automated Driving group,
 Fundaci\'on Tecnalia, Derio, 48160 Spain. Corresponding e-mail: 
 enrique.marti@tecnalia.com.}% <-this % stops a space
\thanks{M. de Miguel and F. Garc\'ia are in Universidad Carloss III de Madrid, 
Legan\'es, 28911, Spain}% <-this % stops a space
\thanks{Manuscript received November 9, 2018; revised Xxxxx 0 , 201x.}}

% note the % following the last \IEEEmembership and also \thanks - 
% these prevent an unwanted space from occurring between the last author name
% and the end of the author line. i.e., if you had this:
% 
% \author{....lastname \thanks{...} \thanks{...} }
%                     ^------------^------------^----Do not want these spaces!
%
% a space would be appended to the last name and could cause every name on that
% line to be shifted left slightly. This is one of those "LaTeX things". For
% instance, "\textbf{A} \textbf{B}" will typeset as "A B" not "AB". To get
% "AB" then you have to do: "\textbf{A}\textbf{B}"
% \thanks is no different in this regard, so shield the last } of each \thanks
% that ends a line with a % and do not let a space in before the next \thanks.
% Spaces after \IEEEmembership other than the last one are OK (and needed) as
% you are supposed to have spaces between the names. For what it is worth,
% this is a minor point as most people would not even notice if the said evil
% space somehow managed to creep in.



% The paper headers
\markboth{Journal of \LaTeX\ Class Files,~Vol.~14, No.~8, August~2015}%
{Shell \MakeLowercase{\textit{et al.}}: Bare Demo of IEEEtran.cls for IEEE Journals}
% The only time the second header will appear is for the odd numbered pages
% after the title page when using the twoside option.
% 
% *** Note that you probably will NOT want to include the author's ***
% *** name in the headers of peer review papers.                   ***
% You can use \ifCLASSOPTIONpeerreview for conditional compilation here if
% you desire.




% If you want to put a publisher's ID mark on the page you can do it like
% this:
%\IEEEpubid{0000--0000/00\$00.00~\copyright~2015 IEEE}
% Remember, if you use this you must call \IEEEpubidadjcol in the second
% column for its text to clear the IEEEpubid mark.



% use for special paper notices
%\IEEEspecialpapernotice{(Invited Paper)}




% make the title area
\maketitle

% As a general rule, do not put math, special symbols or citations
% in the abstract or keywords.
\begin{abstract}
    	After more than 20 years of research, ADAS are common in modern vehicles
        vehicles available in the market. Automated Driving systems, still in
        research phase and limited in their capabilities, are starting early 
        commercial tests in public roads.
        These systems rely on the information provided by on-board sensors, 
        which allow to describe the state of the vehicle, its environment and
        other actors. 
        Sensors are the first stage of any automated driving architecture,
        representing a key factor in the design of the system. 
        This survey reviews existing and upcoming sensor solutions applied to 
        ADAS and Automated Driving, based in both well-known and novel sensing
        technologies, applied to the most common tasks in perception. 
        They are put in context making a historical review of the most
        relevant demonstrations on Automated Driving developed by academy and
        other institutions, focused on their sensing setup. 
        Finally, the article presents a snapshot of the future challenges for
        sensing technologies and perception, finishing with an overview of 
        the commercial initiatives and manufacturers alliances that will show 
        the intention of the market in sensors technologies for Automated
        Vehicles.
\end{abstract}

% Note that keywords are not normally used for peerreview papers.
\begin{IEEEkeywords}
Automated Driving, sensors, perception, OEM.
\end{IEEEkeywords}






% For peer review papers, you can put extra information on the cover
% page as needed:
% \ifCLASSOPTIONpeerreview
% \begin{center} \bfseries EDICS Category: 3-BBND \end{center}
% \fi
%
% For peerreview papers, this IEEEtran command inserts a page break and
% creates the second title. It will be ignored for other modes.
\IEEEpeerreviewmaketitle




\section{Introduction}
\label{sec:01-intro}

Based on recent developments and demonstrations around the world, there is a tendency to think that Automated Driving with a high level of automation will be available in few years. However, there are still some research challenges, such as navigation in urban dynamic environments, accurate obstacle avoidance capabilities, environment understanding in real-time, and perception uncertainties among others. This research is needed to allow cooperative maneuvers between automated and semiautomated vehicles, which still need further efforts in real implementation, specifically in urban environment. 

This paper pays special attention to the sensor and perception stages for automated vehicles. A state of the art, based on the most relevant demonstrators on Automated Driving developed by research institutions and manufactures is presented. This topic has been widely investigated in the literature, but based main on ADAS implementation. It is a key aspect in the future developments highly automated cars, where Real Time Motion planning needs more accurate and robust inputs. In this paper makes an overview of the main problems, application and sensor technologies available in the market.

The rest of the paper is organized as follows. In Section ...


\section{Sensors and technologies}
\label{sec:02-sensors}

This section presents the three principal sensing categories for exteroceptive
perception in Automated Driving: artificial vision, radar and LiDAR.
Exteroceptive sensors and external perception acquire information about the
environment of the vehicle (e.g. road, other vehicles), as opposed to 
proprioceptive sensors that provide information about the state of the vehicle
(speed, accelerations, integrity of components) and are not covered in this 
article. Communications are also out of the scope of the review. 

Next subsections present the advantages, drawbacks and challenges for the three 
types of sensors mentioned above. Each one is followed with a review of
relevant emergent technologies in the field.

In \ref{sec:03-d-information-domains} a taxonomy of information domains is
presented. It is useful for several purposes. First it allows to 
link sensors technologies with perception algorithms described in section
\ref{sec:03-problemsapplications}, since the first provide the raw data needed
by the second. Second, the categorization is used to structure the subsequent
analysis (section \ref{sec:03-e-sensors-for-perception}) about the suitability
and adequacy of the presented sensing technologies for perception in Automated
Driving. This last part includes also the expected performance under different
environmental and weather conditions.

\subsection{Artificial Vision}
Artificial vision is a popular technology that has been used for decades in 
disciplines as mobile robotics, surveillance or industrial inspection. 
It acquires information about objects in the real world by analyzing their
images as captured by photo and video cameras. 
%Cameras are devices that gather light using
%a sensor composed by a grid of thousands or millions of individual detection 
%elements. The amount of light captured by each element is translated into the
%intensity of a pixel in the resulting image. 

This technology offers interesting features, as the low cost of sensors --only 
some types-- and providing range of information types including spatial
(shape, size, distances), dynamic (motion of objects by analyzing their 
displacement between consecutive frames) and semantic (shape analysis).

Artificial vision technology face several challenges, especially in 
applications like automated driving:

\begin{itemize}
    \item Varying light conditions: driving happens at day, at night, indoors, 
    or at dusk or dawn with the sun close to the horizon. 
    Dark spots, shadows, glares, reflections and other effects difficult the
    implementation of reliable artificial visible algorithms.
    
    \item Scenes with a High Dynamic Range (HDR) contain dark and strongly
    illuminated areas in the same frame.
    Most sensor technologies have a limited capacity of capturing both extremes
    simultaneously, so that information is lost in one or the two sides (under- 
    or overexposure). 
        
    \item Low light and high speed: cameras need higher exposures time as
    illumination is weaker. Fast moving elements appear blurred, which can 
    affect later processes as border or feature detection. Also, if the sensor
    does not capture light in its full surface simultaneously (rolling shutter)
    distortion effects can appear in those objects.
\end{itemize}

In \cite{Pueo2016} some of these problems are analyzed from the perspective of
recording scenes in sports.
In order to deal with these difficulties, different technologies and solutions 
have been proposed. 

\begin{itemize}    
    \item High Dynamic Range imaging (HDR): common sensors in photographic and 
    industrial cameras offer a dynamic range of 60-75 dB (10 to 12.5 EVs),
    that is not sufficient in mixed illumination environments as entering or 
    exiting tunnels. Sony launched its IMX390 automotive sensor with an
    extended 120 dB range (equivalent to 20 EVs) and 2k resolution. 
    An automotive grade sensor combining HDR capabilities and 
    Near Infra-Red light detection is analyzed in \cite{Maddalena2005}. 
    In \cite{Strobel2013} a sensor with 130 dB range (global shutter) and up
    to 170 dB (rolling shutter) is presented for industrial safety application.
    
    \item Global shutter sensors and Rolling shutter compensation. In rolling
    shutter cameras the elements of the sensor capture light at different time 
    intervals, creating artifacts as distorted objects under motion or frames
    half-dark-half-light in rooms illuminated by flickering lights (LEDs, 
    fluorescents). 
    These effects have been corrected using software able to compensate scene
    motion vector \cite{Chia-KaiLiang2008}\cite{Chun2008}. Global shutter
    cameras, on the other hand, have the ability to capture light in all the
    elements of the sensor simultaneously. 
        
    \item Captured spectrum: Far infrared cameras (wavelength 900-1400 nm, also
    called thermal cameras) detect the emissions of hot objects including
    living beings. 
    Thermal cameras are effective for pedestrian and animal detection
    \cite{OMalley2008}\cite{Besbes2015} in the dark and through dust and smoke.
    Near Infrared (750-900 nm) complements visible spectrum with a better
    contrast in high dynamic range scenes, improves night visibility. 
    In \cite{Pinchon2018} authors compare visible light, near infrared and far
    infrared cameras in different light and atmospheric conditions.
    
\end{itemize} 

\subsubsection{3D technology}
Traditional camera technology is essentially 2D, but there are some
types of vision sensors that can perceive depth information. This section
describes the three principal types that are already available as commercial
devices, although not always targeting the automotive market.


%\begin{itemize}
%    \item \emph{Stereo vision:} 

\textbf{Stereo vision:} depth is calculated \cite{Hamzah2016} from the 
apparent displacement of visual features in the images capture by two 
monocular cameras pointing in the same direction and separated by some
distance (known as baseline). 
            
One of the greatest advantages of stereo vision systems is their capability 
to provide dense depth maps, as opposed to sparse sensors as LiDARs. 
%Their
%resolution, and maximum and minimum perceived depth are limited by
%camera field of view and imaging sensor resolution.     
Stereo vision drawbacks include issues with low-textured patterns 
(e.g. solid colors) that difficult establishing correspondences between
frames. Also, it has a high computational complexity, and the pair of 
cameras requires a careful calibration to ensure proper translation of 
disparities to depths.

Several good performing monocular SLAM (Simultaneous Location And Mapping)
algorithms \cite{Engel2014}\cite{Engel2018} have been developed in the last
years. 
These systems cannot be classified as stereo but share some of its principles: 
the motion of a single monocular camera setup creates an artificial baseline
between consecutive frames. This allows to estimate depth and camera motion at
the same time.
    
\textbf{Structured light:} a monocular camera coupled with a device that
illuminates the scene with a known pattern of infrared light. 
The distortion of the light pattern when projected over an irregular 
surface is captured by the camera and translated to a depth map.

Structured light devices overcome some limitations of stereoscopic systems
as depending on texture patterns and having a high computational cost. 
However, they require the same high-accuracy calibration \cite{Garbat2013}
and have some additional limitations as short operative ranges (usually 
below 20 meters), limited by the power of the emitter and the intensity of 
ambient light. Reflections can affect its performance.

%    This technology can be combined with stereo vision to improve robustness
%    and operative conditions, as in the commercial device Intel RealSense 
%D4500.

%    \item \emph{Time-of-flight:} 
\textbf{Time-of-flight:} an active sensing technology 
\cite{Hansard2013}, based in the same round-trip-time principle 
of LiDAR sensors (see \ref{sec:02-c-lidar}): an emitter composed of infrared 
LEDs floods the scene
with modulated light that is captured by the sensor after being reflected by 
elements in the environment. 
The round-trip-time can be calculated for each pixel based on the phase shift 
of incoming light, which is then translated to a distance.

Using a non-directed source of light (as opposed to the low divergence laser
emitter in LiDAR) has advantages and disadvantages. 
The advantages include the ability to create dense depth maps, its high 
accuracy and high refresh rate exceeding 50 Hz. The drawbacks include 
problems with intense ambient light and a short operative range (10 to 20 
meters), making this technology unfeasible for many Automated 
Driving applications.

This can change in a short future. Alternative research line as indirect
time-of-flight \cite{Villa2017}, pulsed light time-of-flight or avalanche
photodiodes \cite{Panasonic2018} appear to be close to commercialization. 
These technologies promise ranges between 50 and 250 meters.
    
%\end{itemize}


\subsubsection{Emerging vision technologies}

Event-based vision is a bioinspired technology developed by Zurich University
and the ETH Zurich. The elements of the sensor (pixels) are triggered 
asynchronously and independently when a change on the intensity is detected 
generating a stream of activations also called \emph{events}. 
%An event can be seen as something similar to the output of a feature detection 
%algorithm for artificial vision applications.
Events can be grouped in adaptable time windows for getting a frame-like image,
reaching the microsecond scale for high speed tracking. 
The work \cite{Mueggler2014} shows tracking at 1000 FPS under regular indoor 
lightning conditions. , 
Independence of sensor elements raises the dynamic range of the sensor to 
120 dB. 
Events can be the input to visual odometry \cite{Censi2014} and SLAM
\cite{Vidal2017} applications, relieving the CPU of time consuming operations
on raw images. 
%A steering wheel control for automated driving systems 
%based on deep learning has been shown in \cite{Maqueda2018}.

%Light polarization represents an additional source of information which
%is know to be used by animals able to perceive it (some ants, 
%mantis-shrimp). 
There is an active line of research \cite{Garcia2018} around sensors able to
capture light polarization. It performs better than traditional sensors in
some adverse conditions, and provides information than no other sensor can 
capture.
 

\subsection{Radar}

Radar technology use high frequency electromagnetic waves to measure the
distance to objects based on the \emph{round-trip time} principle, which is the
time it takes the wave to reach the object, bounce on it and travel back to the
sensor. 
Many modern automotive radars use a technique known as digital beamforming
\cite{Hasch2015} to direct the emitted wave in the desired direction.
 
There are two principal technologies: pulsed radar and Frequency-Modulated
Continuous Wave (FMCW). Pulsed radar emits a short burst and measures distance
using the raw round-trip time. FMCW technology emits a signal with a well known
and stable frequency, that is modulated with another continuous signal that
varies its frequency up and down (typically using a triangular shape).
Distance is determined using the frequency shift between the emitted and 
reflected signals. 
Both technologies can exploit Doppler effect to measure the relative speed of 
the target with respect to the sensor. 

%Starting in 2004, EU allocated a permanent 5 GHz wide band around 79 GHz
%\cite{EULawandPublications2004}. Short distance applications as blind spot
%detection, parking assistance, jam assistance and pre-crash measures use the 
%upper part (77-81 GHz), since it offers a better resolution. Long distance 
%applications as ACC use a radar signal around 76-78 GHz. 
%Most modern automotive radars use FMCW technology in multi-frequency chips 
%that 
%can switch between different bands and functions dynamically.

One of the strongest arguments for including radar sensing in automated 
vehicles is its independence of light and weather conditions. 
It works in the dark, and detections are almost equally good with snow, 
rain, fog or dust \cite{Reina2015}. Long range radars can see up to 250 m
in very adverse conditions, where no other sensor works.

Radar sensors present some difficulties and drawbacks:

\begin{itemize}
    \item Sensible to target reflectivity: processing radar data is a tricky
    task, due to the heterogeneous reflectivity of the different materials. 
    Metals amplify radar signal, easing detection of vehicles but increasing
    the apparent size of small objects as discarded cans in the road, while 
    other materials (e.g. wood) are virtually transparent.
    This can cause false positives (detect a non existing obstacle) and false
    negatives (not detecting an actual obstacle).
    
    \item Resolution and accuracy: radars are very accurate measuring distance
    and speed along the line that connects the sensor with a target. However, 
    horizontal resolution depends on the characteristics of the emitted beam.
    Raw angular resolution in digital beamforming systems falls between 2 to 5
    degrees \cite{Schneider2005}, although the application of advanced
    processing techniques can improve angle of arrival accuracy to 0.1 degrees 
    in long range radars (with a narrow FoV around 30 degrees) and 1
    degree for short range sensors \cite{Kissinger2012}. 
    This angular resolution is translated into a spatial resolution that grows
    with distance. At 30 meters it can be difficult to separate (detect as
    independent targets) a pedestrian from a nearby car. 
    At 100 m distance it can be impossible to separate vehicles in neighbor
    lanes, determine if a vehicle is in our same lane, and even if a detection
    is a vehicle or a bridge over the road.
\end{itemize}

\subsubsection{Emerging radar technologies}

One of the most active research area is related with high resolution radar
imaging for automobiles. High resolution radar can extend it use from target 
tracking and speed measuring to providing richer semantic information. 
Recent research explores the use of higher frequency bands to increase
resolution, opening the door to radar imaging and creation of detailed 3D maps. 
An example can be found in \cite{Reina2015}, where a 90GHz rotating radar in
the roof of a car is used to map the environment, including vehicles, static
objects and ground.
The paper \cite{Kohler2013} demonstrates the feasibility of radars operating
between 100 and 300 GHz, analyzing atmospheric absorption and reflectivity of
materials usually found in driving scenarios.
%For example, Arbe Robotics \cite{ArbeRobotics2018} is working in a 
%model with 300 m range, a field of view of 100 degrees horizontal, 30 degrees
%vertical, and a resolution of 1 degree azimuth and 2 degrees elevation. 
%This model is expected to generate a full 4D (3D position plus speed) image of
%the scene at 25-50 Hz, thanks to the embedded machine learning and SLAM
%algorithms.

One of the key technologies that can lead to high resolution radar imaging are 
meta-material based antennas \cite{Brookner2016,Sleasman2017} for efficient
synthetic aperture radars. 
Some manufacturers as Metawave \cite{Metawave2018} are starting 
to offer products oriented to automotive sector based on the technology.

%In a different line, ground-penetrating radar is a technology used long
%time ago in diverse areas ranging from archeology to industrial applications.
%MIT Lincoln Laboratory created a device \cite{Cornick2016} that can be placed 
%below a vehicle to get a reading describing the geological properties of the
%first few meters under the ground. This reading is not affected by snow, 
%water, dust or any other element over the surface of the road. The idea is to
%create maps that can be used later to localize vehicles, with an accuracy of
%a few centimeters. %\cite{XXXX}
%Recently, Wavesense \cite{WaveSense} announced tests in snowy places and is
%close to commercialization stage.


\subsection{LiDAR}
\label{sec:02-c-lidar}
LiDAR (Light Detection And Ranging) is an active ranging technology that 
calculates distance to objects by measuring round-trip time of a laser light 
pulse.
Sensors for robotic and automotive applications use a low power 
near-infrared laser (900-1050 nm) that is invisible, eye-safe. 
Laser beams have a low divergence, so that the reflected power does not decay 
too much with distance. Thanks to this, LiDARs can measure distances up to 
200 m under direct sunlight.
Typically, a rotating mirror is used to change the direction of the laser 
pulse, reaching 360º horizontal coverage. Commercial solutions use an array of 
emitters to produce several vertical layers (between 4 and 128) 
%that, combined with rotation, 
that generate a point cloud representing the environment.
LiDAR sensors feature an extraordinary accuracy measuring distances, averaging
a few millimeters in most cases and degrading to 0.1-0.5 meters in the worst 
cases. This makes LiDAR a good choice for creating accurate digital maps.

However, they have several drawbacks to take into account:

\begin{itemize}    
    \item Low vertical resolution: in low cost models, which usually feature 
    less than 16 layers, vertical resolution (separation between consecutive
    layers) falls down to 2 degrees. At 100 m distance, this is translated into 
    a vertical distance of 1.7 m. High end models reduce this to 0.2-0.4 
    degrees, but at a much higher cost.
    
    \item Sparse measures (not dense): according to \cite{Gatziolis2008} typical
    LiDAR beam divergence is between 0.1 and 1 mrad (0.005 to 0.05 degrees).
    Commercial device Velodyne HDL64 has a 2 mrad divergence \cite{Glennie2010} 
    (0.11 degrees) and a vertical resolution of 0.42 degrees. At 50 meters 
    distance, the 0.3 degree gap is equivalent to a blind strip 0.26 meters
    tall. In low end devices (Velodyne VLP16) this grows to 1.5 meters. 
    Small targets can remain undetected, and structures based on threads and 
    bars are virtually invisible.
    
    \item Poor detection of dark and specular objects. Black cars
    can appear as invisible to the LiDAR, since they combine a color that
    absorbs most radiation with a non-Lambertian material that does not scatter
    radiation back to receiver.
    
%    \item High cost. High-end models have a cost between 25 and 75 k\$.
%    Just a year ago, Velodyne models started at 9000 US\$ for the basic 16
%    layer device, although they have cut prices down to a 50\% to face the
%    recently appeared competitors. Now, some companies are selling
%    equivalent models for less than 4000 US\$. Solid state lidars promise
%    prices an order of magnitude smaller.
    
    \item Affected by dense rain, fog and dust. Infrared laser
    beams are affected by rain and fog because water droplets scatter the light 
    \cite{Wang2008}, reducing the operative range of the device and producing 
    false measures in the front of the cloud. The effect of dust has been
    explored in \cite{Phillips2017}. LiDAR performance in these scenarios is 
    worse than radar, but still better than cameras and human eye.
\end{itemize}

\subsubsection{Emerging LiDAR technologies}
\label{sec:03-lidar-emerging}

%Direct speed measurement is a really useful feature for any sensor. Up to date,
%only radars where able to capture speed, for instance using FMCW signals.

FMCW LiDAR \cite{Nordin2004} emits light continuously to measure objects speed
based on Doppler effect. In the last years some research prototypes suitable for
the automotive market start appearing \cite{Poulton2016}.
%, until recently
%a company named Blackmore announced a commercial version. 
Apart from improving target tracking capabilities, observation of speed can
be useful to enhance activity recognition and behavior prediction, for example 
by detecting the different speeds of limbs and body in cyclists and pedestrians.

%Regarding LiDARs, however, the most popular emerging technology in the last 
%years has been Solid State LiDAR. It offers advantages when it comes
%to operating under strong vibrations and dynamics, apart from being potentially
%smaller, cheaper and faster. 
%However, the market does not offer products combining high resolution and a 
%wide field of view, so mechanical devices are the only option for full 
%360-degree coverage and environment mapping.

Solid state LiDAR is an umbrella term that includes several technologies, two 
of which are oscillating micro-mirrors and Optical Phased Array (OPA).
The first technology combines one or many laser emitters that are directed
with micro-mirrors that can rotate around two axes, so that the beam 
can be directed within a cone. Manufacturer LeddarTech commercializes devices
based on this technology \cite{LeddarTech2016}.
Optical phased arrays \cite{McManamon1996} is a technology similar to that used 
for EBF radars. 
%An array of optical emitters generate coherent signals with a well
%controlled phase difference. This generates a far-field radiation pattern 
%pointing in a direction that depends on the phase. 
that allows to control the direction of the beam with high accuracy and speed.
Quanergy \cite{Eldada2017} is one of the few manufacturers commercializing
devices based on this technology.
% with a focus on the automotive sector, its S3
%model features a 120 degrees FoV with a range of 150 m. 
%They report a spot size of 5 cm at 100 m, that is  comparable to a beam
%divergence of 0.5 mrad --4 times smaller than Velodyne HDL64 model.

OPA technology has additional advantages over mechanical LiDAR: the scan 
pattern can be random in the entire FoV, which is great for characterizing fast 
moving objects. It is possible to observe only a region of interest within the 
FoV. Also, it is possible to augment the point density within each frame for 
better resolution. The three features can be combined to do fast low resolution 
inspection of the full FoV, and then tracking with high resolution the objects
of interest for enhanced shape recognition even at far distances.
%This is similar to Waymo's claims about the LiDARs they have developed for 
%their self-driving vehicles, as described in section 
%\ref{sec:04-relevantdemos}.

\subsection{Relevant information domains}
\label{sec:03-d-information-domains}

The task of a perception system is to bridge the gap between sensors providing 
data and decision algorithms requiring information.
A classical differentiation between both terms is the following: data is 
composed by raw, unorganized facts that need to be processed. 
Information is the name given to data that has been processed, organized, 
structured and presented in a proper context.

The following taxonomy (table \ref{tab:info-taxonomy}) is tightly related with 
the goals of perception stage (covered in section
\ref{sec:03-problemsapplications}). It allows to present conclusions about the
suitability of each sensor technology for the different perception tasks in a
clear and organized way.

\begin{table}%[H]
    \caption{Information taxonomy in Automated Driving domain}
    \label{tab:info-taxonomy}
    \begin{tabular*}{\linewidth}{lr p{5.5cm}} %{\textwidth}{lrL}
        \hline %\toprule
        \textbf{Category} & \textbf{\#}	& \textbf{Information type}	\\
        \hline %\midrule
        \multirow{2}{*}{Ego-vehicle}
        & 1 & Kinematic/dynamic (includes position) \\
        & 2 & Proprioceptive (components health/status) \\
        \hline %\midrule
        \multirow{3}{*}{Passengers/driver}
        & 3 & Driver awareness/capacities \\
        & 4 & * Driver intentions (mind model)  \\
        & 5 & Passenger status (needs, risk factors) \\
        \hline %\midrule
        \multirow{4}{*}{Environment}
        & 6 & Spatial configuration: location, size, shape, fine features 
        \\
        & 7 & Identification: class, type, identity \\
        & 8 & Regulation and semantics: traffic signs, road marks, other 
        elements \\
        & 9 & Contextual factors: weather, driving situation(e.g. jam, 
        highway, off-road) \\
        \hline %\midrule
        \multirow{4}{*}{External actors}
        & 10 & Spatial features: location, size, shape, fine features  \\
        & 11 & Kinematic/dynamic: position, motion \\
        & 12 & Identification: class, type, identity \\ 
        & 13 & Semantic features: vehicle lights, pedestrian clothes, gestures 
        \\
        & 14 & * Situational engagement: collaborative/aware 
        (adult pedestrians, other vehicles) vs non-collaborative/unaware 
        (animals, children) \\ 
        \hline %\bottomrule
    \end{tabular*}
\end{table}

Elements with an asterisk are derived information. This is, they that can 
be inferred from sensed data but not directly observed. It is mostly related 
with internal state of external entities, as the intentions of human beings and 
animals.

\subsection{Using sensors for perception}
\label{sec:03-e-sensors-for-perception}

Sensor selection and arrangement is one of the most important aspects in the 
design of a perception system for automated vehicles. It also has a great impact
in its cost, with some setups having several times the price of the rest of 
the vehicle. 
Many analysis focus in spatial coverage and range, but this epigraph summarizes
two other aspects of the uttermost importance: type of information acquired and 
impact of environmental factors.

In first place, the characteristics of a sensing technology determines its 
suitability for acquiring certain types of information, and restricts its range 
of operative conditions.
Figure \ref{fig:information_vs_sensors} relates the principal sensing 
technologies currently used in the automotive market and Automated Driving
initiatives with relevant types of information identified in Table 
\ref{tab:info-taxonomy}. The adequacy of a sensor for acquiring a certain type
of information (or equivalently, the expected quality of that type of
information when captured by that sensing technology) is classified in three
levels: Good (green shading, tick), Medium (ambar shading, letter M) and Bad
(red shading, letter B).

\begin{figure}[h]
    \centering
    \includegraphics[width=0.95\linewidth]{"img/information_types_sensors"}
    \caption{Sensor adequacy for relevant types of information}
    \label{fig:information_vs_sensors}
\end{figure}

Sensors and perception are expected to work uninterruptedly during vehicle 
operation. Weather and other environmental factor can degrade sensor
performance, but each technology is affected in a different way. 
Figure \ref{fig:sensors-environ} summarizes the effect of common external
factors in the performance of the analyzed sensing technologies, using the
same notation as Figure \ref{fig:information_vs_sensors}.

\begin{figure}[h]
\centering
\includegraphics[width=0.68\linewidth]{"img/sensors_atmospheric_conditions"}
\caption{Sensor robustness under atmospheric and environmental factors}
\label{fig:sensors-environ}
\end{figure}

%A perception system needs to cover adequately the relevant surroundings of the
%vehicle. Ideally, this includes 360 degrees around the vehicle up to several
%hundreds meters. Figure \ref{fig:range-fov} shows usual operative range and
%field of view of relevant sensing technologies. 
%
%\begin{figure*}[h]
%    \centering
%    \includegraphics[width=0.7\textwidth]{"img/plot_range-fov"}
%    \caption{Range-FoV for depth sensors (3D sensing technology)}
%    \label{fig:range-fov}
%\end{figure*}



\section{Problems and applications}
\label{sec:03-problemsapplications}

This section analyzes the state of the art in perception systems for Automated
Driving. A set of behavioral competences is identified, followed
by a systematic literature review that analyzes the 
solutions for each category, organized by sensor technology.

\subsection{Behavioral competencies}

Behavioral competencies in Automated Driving ``refers to the ability of an 
Automated Vehicle to operate in the traffic conditions that it will regularly
encounter" \cite{Nowakowski2015}. The NHTSA defined a set of 28 core 
competencies for normal driving \cite{NHTSA2016}, that have been augmented to a 
total of 47 by Waymo \cite{Waymo2017} in their internal tests.
Table \ref{tab:behavioral-competences} selects a subset of those behavioral
competencies and arranges them in categories that are used to structure the state of the art in
perception algorithms in a purpose oriented approach.

\begin{table*}[t] %[H]
    \caption{Behavioral competences and relation with information taxonomy 
        (see Table \ref{tab:info-taxonomy})}
    \label{tab:behavioral-competences}
    \begin{tabular*}{\textwidth}{m{4cm} l p{11cm}}%{XlL}
        \hline %\toprule
        \textbf{Competence}	& \textbf{Information type} & \textbf{Behavior}	
        \\
        \hline %\midrule
        \multirow{4}{4cm}{Automatic Traffic Sign Detection
            and Recognition (TSDR)}
        & 8    & Detect Speed Limit Changes, Speed Advisories, Traffic Signals 
        and Stop/Yield Signs \\
        & 8    & Detect Access Restrictions (One-Way, No Turn, Ramps, etc.) \\
        & 8    & Detect Temporary Traffic Control Devices \\
        & 6, 8 & Detect Passing and No Passing Zones  \\
        \hline %\midrule
        \multirow{4}{*}{Perception of the environment}
        & 8 & Detect Lines \\
        & 6, 8 & Detect Detours  \\
        & 6 & Detect faded/missing roadway markings, signs and other 
        temporary changes in traffic patterns \\
        & 9 & Perception under weather or lighting conditions 
        outside 
        vehicle’s capability (e.g. rainstorm) \\
        \hline %\midrule
        \multirow{6}{4cm}{Vehicles, pedestrians and other obstacles 
            detection}
        & 10, 12, 13 & Detect Non-Collision Safety Situations (e.g. vehicle 
        doors ajar) \\
        & 10, 11, 12, 13 & Detect Stopped Vehicles, Emergency Vehicles, Lead 
        Vehicle, Motorcyclists, School Buses \\
        & 6(, 1)  & Detect Static Obstacles in the Path of the Vehicle \\
        & 6, 8, 9, 10, 11, 12 & Detect Pedestrians and Bicyclists at 
        Intersections, Crosswalks and in the Road. \\
        & 10, 11, 12 & Detect Animals \\
        & 10, 12, 13 & Detect instructions from Work Zones and People 
        Directing Traffic in Unplanned or Planned Events, Police/First 
        Responder Controlling Traffic, Construction Zone Workers Controlling, 
        Citizens Directing Traffic After a Crash (Overriding or Acting as 
        Traffic Control Device) \\
        
        \hline %\bottomrule
    \end{tabular*}
\end{table*}

This set of competences represents the link between perception and decision
(planning), as a counterpart to the information taxonomy presented in the
previous section (Table \ref{tab:info-taxonomy}), which linked sensors and 
perception algorithms. 
Both tables can be combined to evaluate the suitability of sensor technologies
for creating some set of Automated Driving capacities.

The next subsections describe the state of the art in perception techniques for
the three identified categories of behavioral competencies.

\subsection{Automatic Traffic Sign Detection and Recognition (TSDR)} 
%[Definition? ]. 
Traffic signs are visual devices with a well defined aspect, that transmit a 
clear and precise piece of information about traffic regulation, warnings about
factors affecting driving and other informative statements. The spatial and
temporal scopes of applicability are also defined in the sign, either 
explicitly or implicitly.
Acquiring information from road traffic signs involves two major tasks: 
Traffic Sign Detection (TSD) which consists on finding the location, 
orientation and size of traffic signs in natural scene images, and Traffic Sign 
Recognition (TDR) or classifying the detected traffic signs into types 
and categories in order to extract the information that they are providing to 
drivers.
%Automatic TSDR has two different applications: Real time detection and
%recognition is used in ADAS or autonomous driving, and automatic road traffic 
%sign mapping systems are used for generating a database of traffic signs 
%of a certain area. This last application does not need to work on real-time. 

%Traffic signs are designed for human visual perception. Two sensors are mostly
%used for these tasks: monocular cameras in different configurations 
%(single camera, multiple focal or multiple cameras) and LiDAR sensors.
Below are shown the most relevant solutions according to the type of sensor 
and the technology used.

\subsubsection{Camera based solutions}
Cameras are the most common sensor for TSDR. They can be used for TSR, TSD or both at the same time.
%As an example of TSR, \cite{frejlichowski2015application} proposes a method 
%based on the Polar-Fourier Grayscale Descriptor, which applies the information 
%about silhouette and intensity of an object. In \cite{gao2015learning} a 
%learning method based on a histogram intersection kernel is used to quantize 
%features, that are encoded in a look-up table.
As an example of TSR, \cite{frejlichowski2015application} proposes a method 
based on the Polar-Fourier Grayscale Descriptor, and \cite{gao2015learning} a 
learning method based on a histogram intersection kernel.
%For TSD, \cite{zhang2017real} proposes a method based on a fast Convolutional 
%Neural Network (CNN) inspired in the YOLOv2 network. This algorithm 
%can detect the position of the traffic sign and classify it into Mandatory 
%(blue colored), Danger (triangle shaped) and Prohibitory (red circle). 
%\cite{villalon2017traffic} detects stop and yield signs with a statistical 
%template built using color information in different color spaces (YCbCR and ErEgEb).
For TSD, \cite{zhang2017real} proposes a method based on a fast Convolutional 
Neural Network (CNN) inspired in the YOLOv2 network. This algorithm 
can detect the position of the traffic sign and classify it according to its shape. 
\cite{villalon2017traffic} detects stop and yield signs with a statistical 
template built using color information in different color spaces (YCbCR and ErEgEb).
TSD techniques can also be applied to traffic light detection, as in 
\cite{hosseinyalamdary2017bayesian}, where a Bayesian inference framework to 
detect and map traffic lights is described. A different approach is proposed by 
\cite{gu2011traffic} that uses a dual focal camera system composed of a wide 
angle camera and a telephoto camera which is moved by mirrors 
in order to get higher quality images of the traffic signs.
%Camera sensors can also perform TSD and TSR tasks as is 
%shown in the following works. \cite{miyata2017automatic} uses local binary 
%pattern method for detecting speed signals and a neural network for the 
%recognition of the numbers of the sped limit sign. \cite{yang2016towards} 
%presents a fast detection method based on traffic sign proposal extraction and 
%classification built upon a color probability model and a color Histogram of
%Oriented Gradients (HOG) combined with a convolutional neural network to
%further classify the detected signs into subclasses.
%\cite{wali2015automatic} performs detection and recognition tasks using a RGB 
%colour segmentation and shape matching followed by support vector machine (SVM) 
%classifier. 
Camera sensors can also perform TSD and TSR tasks as is 
shown in the following works where first the signals are detected attending to
their color or shape, and then they are classified using machine learning techniques (CNN or SVM)
\cite{miyata2017automatic, yang2016towards, wali2015automatic}. In \cite{timofte2014multi}
a system composed by eight roof-mounted cameras which takes images every meter
perform offline TDSR to create a database with more than 13,000 traffic signs annotations


\subsubsection{LiDAR based solutions}
LiDAR sensors have been used for TSD. Their 3D perception capabilities are 
usefult to determine the position of the sign and its shape, and can also use 
the intensity of reflected light to improve detection accuracy based on the
high reflectivity of traffic signs. \cite{gargoum2017automated} 
performs detection in three steps: first the point cloud is filtered by 
laser reflection intensity, then a clustering algorithm is used to detect 
potential candidates, followed by a filtering step based on the lateral 
position, elevation and geometry that extracts the signs. 
\cite{weng2016road} goes one step further and makes a primary 
classification attending to the sign shape (rectangular, triangular and 
circular).

\subsubsection{Sensors Fusion solutions}
A system that combines LiDAR and Cameras can improve the sign detection and 
recognition as it has the advantages and the information of both sources. 
\cite{zhou2014lidar} trains a SVM with 10 variables: 9 of different color 
spaces provided by the camera (RGB, HSV, CIEL*a*b*) plus reflection intensity 
observed by LiDAR. After verifying the 3D geometry of detected signs, a linear
SVM classifier is applied to HOG features.
\cite{guan2018robust} method detects traffic signs in LiDAR point clouds
using prior knowledge of road width, pole height, and traffic sign reflectance, 
geometry and size. Traffic sign images are normalized to perform classification 
based on a supervised Gaussian–Bernoulli deep Boltzmann machine model.


%Como el resto no tiene un summary lo quito

%\subsubsection{Summary of sensors and their use for TSDR}
%\begin{itemize}%[leftmargin=20mm,labelsep=5.8mm]
%\item \textbf{LiDAR:} Filtering for intensity (traffic signal has high Reflectance), 3D position and shape. Clustering for sign candidates
%\item \textbf{Camera}
%\subitem \textbf{Monocular B/W:} Feature extraction.
%\subitem \textbf{Monocular Color:} Color information. Classification with 
%different methods: SVM, CNN, ..., or both, detection and classification.
%\subitem \textbf{Multiple cameras:} Mapping signals in roads, 3d position.
%\subitem \textbf{Multiple focal:} Short focal for detection and long focal for 
%higher quality sign image.
%\item \textbf{Fusion LiDAR and Camera:} Using LiDAR intensity and 3d position 
%and shape information for detecting, and camera color and shape information for 
%recognizing
%\end{itemize}

\subsection{Perception of the environment}
The purpose of this competence is to characterize and describe the road, which
represents the most direct piece of environment of a vehicle. This 
involves two different aspects: characterize road surface geometry and detect road marks (lanes
and complements traffic signs as stops, turns or stopping lines).

%) and the lane marks that delimit 
%those lanes. This information can be used in real-time, or can be combined with
%a GNSS-based high accuracy localization device to generate a detailed road map
%that can be used by other vehicles.
%The second task is related to the localization of the vehicle in the road. It 
%can be either on map generated by the first map, or in unknown roads. 

%Reliability is a critical factor for this competence, since missing a mark and
%false detections can cause an Automated Vehicle to change lanes inadvertently 
%or even go out of the road. Guaranteeing the quality of mark detection is a 
%challenging task not only due to varying lane marks (heterogenous widths, 
%faded marks), but also due to weather and environmental conditions: 
%Different light intensity (sunny day, night, shadows), different 
%visibility conditions (fog, snow, rain) and different reflectivity of the road
%(wet or dry asphalt). 

Road marks, as traffic signs, are designed to be detected and 
correctly interpreted by human drivers under a wide variety of external 
conditions. This is achieved using reflective painting and high contrast 
colors. Cameras and less frequently LiDARs have been used for detecting them.
Road geometry description has been approached using cameras, LiDARs and radars.

In the following lines, the most relevant works about this topic are presented, 
organized by the type of sensor they use.

%Semantic representation of the lanes
%Using the reflectivity information recorded by lidars enable the our system to detect lane markers in the presence of shadows, against direct sunlight and even at night

\subsubsection{Camera based solutions}
can be grouped in three categories depending on the specific sensor 
configuration.

\textbf{Single Monocular.}
Using only one 
camera looking at the road in front of the vehicle it is possible to estimate 
its shape and lanes, the position of the vehicle in the road and  
detect road marks. A survey of the most 
relevant algorithms used for this purpose, mainly for camera sensors is 
presented in \cite{hillel2014recent}.

\textbf{Multiple Monocular cameras.} 
Some works \cite{lee2017avm, kum2013lane} arrange multiple cameras 
around the vehicle (typically four, one on each side) to get 360-degree 
visual coverage of the surroundings. 
A different configuration is used in \cite{Ieng2003}, where two lateral cameras
are used to localize the vehicle. 

\textbf{Binocular or Stereo.} 
The main advantage of binocular cameras is their 3D perception capabilities.
It makes possible to detect the ground plane and road boundaries 
\cite{schreiber2013laneloc, ozgunalp2017multiple}, improving road mark 
detection. 

\subsubsection{LiDAR based solutions}
Main application of LiDARs in road perception is related with detecting the 
ground plane and road limits, as well as detecting obstacles that could occlude 
parts of the road.
In recent works, LiDAR based solutions also take advantage of the higher 
reflectivity of road marks with respect to the pavement (gray and black 
material) to detect lane \cite{yang2012automated, li2013new} and
pavement markers \cite{Zhang2016}.
Poor road maintenance can affect markers reflectivity to the point of making 
them undetectable by LiDAR. This can be solved by fusing LiDAR 
data with cameras able to perceive non reflective lane marks \cite{lee2017avm}.
Some works use a 2D LiDAR sensor to extract road geometry and road marks 
\cite{nie2012camera, kim2015lane}.

%In some works LiDARs are used to detect road boundaries and map them into images in order to get training data for segmentation algorithms Used to classify images. \cite{}

\subsubsection{Radar based solutions}
%The main advantage of Radar sensors is their ability to work in all weather 
%conditions (darkness, rain, fog, snow, etc). But due to the poor image 
%contrast 
%of these types of sensors, the only usefull information for that competence 
%that can be extracted is the position of the obstacles (in order to know 
%oclusions of the road) and the road itself. 
Radars have been used to determine road geometry based on the principle that the
road acts as a mirror for the sensor, returning a very small amount of the 
emitted power, while the sides of the roads return a slightly higher 
amount of power. Road limits have been estimated with a
maximum error of half a lane at zero distance from the host vehicle and less 
than one lane width at 50 meters distance. This information can be fused with
camera images to improve both detections 
\cite{kaliyaperumal2001algorithm, ma2000simultaneous, Janda2013}.

\subsection{Vehicles, pedestrians and other obstacles detection}
%This section reviews the use of different sensors for detecting other elements 
%in the road, including vehicles, pedestrians and any other kind of obstacle 
%that may appear such as motorcycles, bicycles, animals, etc. 
%The advantages and disadvantages of sensors for this application are discussed.
This competence involves moving elements that can be in the path of the 
vehicle, so it requires extracting more information. Apart from detection and 
classification, it is also important to determine the position of obstacles 
with respect to the vehicle, their motion direction, speed, and 
future intentions when possible. 
This information will be the input to other systems like path planners 
or collision avoidance systems (reviewed in \cite{mukhtar2015vehicle}).

\subsubsection{Camera based solutions}
Different configurations have been used for camera based obstacle detection, 
includinng single monocular camera, multiple cameras, stereo cameras and 
infrared cameras.

Cameras can be placed in different locations. The front of the vehicle is 
the most common placement since the most critical obstacles will be in front of 
the vehicle, but many works explored other positions in order to increase the
FoV. 
A camera placed on the side-view mirror, in the 
passengers window \cite{chang2008real} or looking backwards \cite{liu2007rear}
can detect and track vehicles trying to overtake the ego-vehicle, 
improving the decision of lane change maneuvers \cite{alonso2008lane, 
song2007lateral, blanc2007larasidecam}. 
An omnidirectional camera mounted
on the top of the vehicle has been used in \cite{gandhi2006vehicle}
to detect obstacles and estimate ego-motion.

Stereo cameras are widely used for obstacle detection as they provide 3D 
information of the position of the obstacles. A large review of the different 
algorithms used for this kind of cameras can be found in \cite{bernini2014real}.
FIR cameras are independent of scene illumination and can spot obstacles at
night \cite{olmeda2013pedestrian}. Relevant moving elements (vehicles, 
pedestrians, animals) are usually hot and, thus, easy to detect with FIR 
cameras. However, this sensor has to be complemented with other technologies
as in \cite{krotosky2007color}, since cold obstacles like parked vehicles or
trees can be not perceived.
\cite{sivaraman2013looking} presents and explains in detail several camera
solutions and the algorithms used for detection.

\subsubsection{LiDAR based solutions}
LiDAR technology allows to detect and classify surrounding elements, providing
a very accurate 3D position and its shape. 
As it is an active sensor its performance is not affected by the illumination 
of the scene, so it can work also at night. Several approaches for LiDAR 
obstacle detection are shown in \cite{li2016vehicle}.

\subsubsection{Radar based solutions}
The primary use of automotive radars is detection and tracking of other 
vehicles on the road, thanks to their high accuracy measuring target 
distances and relative speed, long range detection and performance in adverse 
weather conditions \cite{blanc2004obstacle}. 
Because of its low resolution radar detections are
usually fused with other sensors as cameras \cite{garcia2012data} or, in some
works, with LiDAR \cite{gohring2011radar}.

%\textbf{Microphone based solutions:}
%Microphones sensors can be used to detect aproaching vehicles from the rear 
%side, an array of multiple microphones placed in the surroundings of the 
%vehicle can estimate even the direction of these aproaching vehicles 
%\cite{mizumachi2014robust}.

\subsubsection{Multiple sensors fusion solutions}
This competence requires estimating a large number of variables simultaneously,
creating difficulties for any single sensor solution. This is a good scenario
for sensor fusion systems, that can combine the strengths of each sensor to
improve the solution. 
%Almost every possible combination have been tested. 

%The most common combinations are described in the following lines.
Radar and LiDAR fusion \cite{gohring2011radar} increases the precision of 
the speed obtained only with LiDAR and keeps a good position and speed 
estimation quality when radar is unavailable (especially in curvy roads).
Radar and vision fusion techniques use radar information to locate areas of 
interest on the images, which are then processed to detect vehicles and improve 
their position estimation \cite{alessandretti2007vehicle}.
LiDAR and vision sensors are fused in \cite{premebida2007lidar}. Obstacles
are detected and tracked with the LiDAR, and the targets are classified using
a combination of camera and LiDAR detections.



\section{Relevant works and demos}
\label{sec:04-relevantdemos}
This section describes some of the most relevant technological demonstrations, 
competitions, challenges and commercial platforms related with Automated 
Driving, starting from pioneering works in late 1980s until present day. Figure 
\ref{fig:tech-demos} arranges them in a timeline, with the focus on the sensors 
equipped by each platform.

The timeline allows to discern different stages (``ages'') in the development 
of Automated Driving technology, and to identify trends and approaches from the 
perception point of view for Automated Vehicles.

\begin{figure}[p] %[h]
  %\centering
  \includegraphics[width=0.95\textheight,angle=90,keepaspectratio]{"img/AD_demos_Timeline"}
  \caption{Timeline: relevant AD demos and their exteroceptive sensor 
  setup}
  \label{fig:tech-demos}
\end{figure}

\subsection{Pioneer works (1980-2000)}

Pioneer works in Automated Driving starts around mid-1980s focused in vision 
based techniques, which represented a huge computational burden for the
embeddable computers of the time. Automated Vehicles VaMoRs 
\cite{Dickmanns1987} and VaMP \cite{Gregor2002} from Bundeswehr 
University of Munich used a saccadic vision system: cameras on a rotating 
platform that focus in relevant elements. 
The University of Parma started its project ARGO in 1996. % \cite{Broggi1998}.
The vehicle completed over 2000 km of autonomous driving in public roads 
\cite{Broggi1999}, using a two camera system for road following,
platooning and obstacle avoidance. 
%Sixty transputers executed an intelligent 4-D approach to object tracking,
%delivering a huge amount of computational power according to the standards at
%that moment.

%In early 1990s INRIA creates the concept of a "Cybercar", a small automated 
%electric vehicle for shared urban use \cite{Parent1993}. Project Praxitèle
%\cite{Massot1999} (a mixed public and industry initiative) led to a fully
%functional prototype. 
The Cybercar concept is born in early 1990s \cite{Parent1993}
as an urban vehicle with no pedals or steering wheel. 
In 1997 a prototype is installed in Schippol airport to transport passengers 
between terminal and parking \cite{Ozguner2007}. It used a LiDAR and vision
system to drive automatically in a dedicated lane with semaphores and
pedestrian crossings.

Also in 1997, the National Automated Highway System Consortium presented a 
demonstration of Automated Driving functionalities \cite{Thorpe1997}, intended 
to be a proof of technical feasibility. 
The demo showed road following functionality based on vision sensors, distance 
maintenance based on LiDAR, vehicle following based on Radar and other 
functionalities including cooperative maneuvers and mixed environments.

%At that time, relevant functional demonstrators strongly relied on visual 
%processing. 
%The University of Parma started in 1996 a project with its vehicle 
%ARGO \cite{Broggi1998}, equipped with two cameras that allowed road following,
%platooning and obstacle avoidance. The vehicle completed over 2000 km of 
%autonomous driving in public roads \cite{Broggi1999}.
%They detected lane lines and 
%vehicles using classical image processing techniques including preprocessing 
%steps, feature extraction, model fitting and spatio-temporal filtering.


\subsection{Proof of feasibility (2000-2010)}

In year 2004 DARPA started its Grand Challenge series to foster development of
Automated Driving technologies. The achievements over those three years 
not only represented a huge leap forward, but also called the attention of
powerful agents.
Two first challenges (2004 and 2005) consisted in covering a route over dirt
roads with off-road sections, with a strong focus in navigation and control.
Stanford University won the 2005 edition, equipping its vehicle Stanley with
5 LiDAR units, a frontal camera, GPS sensors, an IMU, wheel odometry and two 
automotive radars \cite{Thrun2006}. 
The Urban Challenge (2007) changed the focus to interaction with other vehicles,
pedestrians and obeying complex traffic regulations. Carnegie Mellon University
team ended in first position with its vehicle Boss 
%\footnote{http://www.tartanracing.org/press/boss-glance.pdf} 
\cite{TartanRacing2005, Urmson2007}, 
featuring a perception system composed by two video cameras, 5 radars and 13
LiDAR (including a roof mounted unit of the novel Velodyne 64HDL).
%A cluster of 10 server blades processed a complex behavioral model
%\cite{Urmson2007} for covering all the expected situations.


%In 2004, DARPA started its series of Grand Challenges to foster the development
%of robotics technology. 
%The first edition ended without a declared winner, since no contestant manage 
%to
%cover even a 10\% of the 240km route including dirt roads and off-road 
%sections.
%In 2005 edition, five of the 23 contestants finished the 212 km race. 
%The winner was Stanford University racing team. Its vehicle Stanley carried 5 
%LiDAR units 
%%used to create a 3D map of the environment with special attention to road 
%%geometry 
%, a frontal camera, GPS sensors, an IMU, wheel odometry and two 
%automotive radars \cite{Thrun2006}. These sensors were the input of a 
%complex processing pipeline distributed over 6 computers, that involved
%perception and estimation techniques, artificial intelligence, 3D mapping, 
%risk assessment and path planning.
%%, requiring 6 computers to do the processing. 
%
%For the next DARPA Challenge (Urban Challenge, 2007), participants had to
%complete a 96 km course in urban area, sharing the road 
%with other participants and cars driven by professional drivers. Robots were 
%required to obey traffic regulations, avoid obstacles and negotiate 
%intersections properly. Carnegie Mellon University team won the contest with 
%its vehicle Boss \cite{TartanRacing2005}, featuring a complex perception
%system composed by two video cameras, 5 
%radars and 13 LiDAR (including a roof mounted unit of the novel Velodyne 
%64HDL).
%A cluster of 10 server blades processed a complex behavioral model
%\cite{Urmson2007} for covering all the expected situations.

These events triggered the attention of Google. 
The company hired around 15 scientists from the DARPA challenge, 
including the winners of 2005 and 2007 \cite{Montemerlo2008}, 
\cite{Levinson2011}. Google's (and Waymo's) approach to self-driving vehicles
is largely founded in LiDAR and 3D mapping technologies \cite{Chapell2016}. 
All their vehicles have had a roof-mounted spinning LiDAR: Toyota Prius (2009),
the Firefly prototype (2014) and Chrysler Pacifica (2016-present).

The University of Parma created the spin-off VisLab in 2009. 
They are strong supporters of artificial vision as the main component of 
perception systems for AD. 
In 2010 they completed the VisLab Intercontinental Autonomous 
Challenge (VIAC): four automated vans drove from Italy to China over public 
roads that included degraded dirt roads and unmapped areas \cite{Bertozzi2011}.
The leading vehicle did perception (with cameras and LiDARs), decision and 
control, with some human intervention for selecting the route and managing 
critical situations \cite{Broggi2012}. 
%The 13,000 km long trip included
%unmapped areas, degraded dirt roads and different traffic conditions. 
In 2013 the PROUD test put a vehicle with no driver behind the wheel in Parma 
roads for doing urban driving in real traffic \cite{Broggi2013}. 
%Selected sensor configuration was very similar to that used in VIAC three years
%before.
 
\subsection{Race to commercial products (2010-present)}
 
In the last decade the landscape of Automated Driving has been dominated by 
private initiatives that foresee the coming of Level 4 and 5 systems in a few 
years. This vision gave birth to several companies devoted to this end, most of 
which were founded by people coming from the DARPA experience, or hired them to
lead the project \cite{Chapell2016}. 

Examples include the nuTonomy (co-founded
by the leader of the MIT team in 2007 Challenge), Cruise (founded by a 
member of the same team), Otto (founded by a participant in 2004 and 2005 
Challenges), Uber (hired up to 50 people from the CMU Robotics Lab),  
Zoox robotaxi company (co-founded by a member of the Stanford 
Autonomous Driving team) %with an expertise in LiDAR automated calibration 
\cite{Levinson2011a}, and Aurora (similar story with people from 
Uber, MIT and Waymo \cite{Anderson2013}).

Car manufacturers reacted a bit slower. 
%Stanford and Audi teamed to complete an
%automated ascent to Pikes Peak in 2010 (focused on control and not in
%perception) \cite{Funke2012}, they have not really entered into scene
%until the last five years. 
Some of them started independent research lines, for example
BMW has been testing automation prototypes in roads since 2011 
\cite{Aeberhard2015a} and Mercedes-Benz Bertha project \cite{Ziegler2014}
%\cite{Bender2014} 
drove in 2013 a 103 km route in automated
mode using close-to-market sensors (8 radars and 3 video cameras)
%This work presented new concepts as the \emph{Lanelets} for road 
%representation 
% and other innovative solutions .
% processed in a heterogeneous computing platform (FPGAs, embedded 
%processors).
, but in the end most manufacturers have created coalitions with technological
startups as enumerated in section \ref{sec:oem-ad}.


%This approach to perception --avoiding LiDARs. 
%as expensive and far from mass 
%production devices-- has been supported by other companies. As an example,
%Mobileye started working in embedded computer vision devices in 1999, and by
%2015 their technology was present in more than 25 car brands. Some years ago
Mobileye
%\footnote{https://www.mobileye.com/our-technology/} 
started working in a vision-only approach to Automated Driving
%\cite{Mobileye2018} 
a few years ago. 
%Their AI is claimed to hand the car with an ``\emph{assertive driving style}'' 
%that deals with traffic in a much less conservative way than its competitors,
%while being safe \cite{Shalev-shwartz2016, Shalev-Shwartz2017}. 
After testing in real conditions \cite{Edelstein2018}, they presented a demo 
with an automated Ford equipped just with 12 small monocular cameras for fully 
Automated Driving in 2018 \cite{Scheer2018}.

Tesla entered the Automated Driving scene in 2014.
All their vehicles were equipped with a monocular camera (based on 
Mobileye system) and an automotive radar that enabled the Level 2-3 AutoPilot
functionality. 
%that gathered data for training
%a "ghost" self-driving system based on reinforcement learning. 
Starting 2017 new Tesla vehicles include the ``version 2'' hardware, 
composed by a frontal radar, 12 sonars, and 8 cameras.
%, together with a nVidia Drive PX2 processing platform.
This sensor set is claimed to be enough for full Level 5 Automated Driving
\cite{Hawkins2017}, which will be available for a fee (when ready) through a
software update.

In 2015 VisLab was acquired by Ambarella, a company working on low power chips
able to process high resolution dense disparity maps from stereo cameras
\cite{Ambarella2018}. 
Its latest demo \cite{AUVSI2018} fused data from 10 stereo pairs into a
ultra-high resolution 3D scene delivering 900 million points per second.
Long range vision mix a forward facing 4k stereo pair with a radar for better
performance under low light or adverse weather conditions. 
%This is a different approach to visual-based perception, 
%since stereo vision aims to get the best of both LiDARs and image processing 
%in 
%a single tool, with additional advantages.

Delphi Automotive completed in 2015 an automated trip between San Francisco and
New York city using a custom Audi Q5 with 10 radars, 6 LiDARs and 3 cameras 
onboard. In 2017 they acquired nuTonomy (the first company to deliver a 
robotaxi service in public roads) and created Aptiv. 
Aptiv presented an automated taxi for CES conference in january 2018, as part
of a 20 vehicle fleet that has been serving a set of routes in Las Vegas for
some months. The taxis have an extensive set of 10 radars and 9 LiDARs embedded 
in the bodywork, plus one camera.

Meanwhile, Waymo has grown a fleet of Chrysler Pacifica minivans that has
self-driven 10 million miles by october 2018. Their efforts have reportedly 
cut prices of LiDAR sensors to less than one tenth in a few years. 
They claim to have created two ``new categories of LiDAR'' \cite{Waymoteam2017} 
in the way, one for close range perception including below the car, and the
other for long range. The long-range LiDAR can reportedly zoom dynamically into 
objects on the road, letting the vehicle see small objects up to 200 m away. 
This reminds the features of OPA solid state LiDARs (see section
\ref{sec:03-lidar-emerging}): random sampling across the 
scanning area and adaptive resolution.

% (which still can be high, since the cost of the
%64-layer Velodyne was over US\$ 75,000 when Waymo started its experiments).


\section{Discussion}
\label{sec:06-discussion}

To conclude the survey this section makes a discussion of the future challenges that new autonomous vehicles will need to cope with, technical and implantation ones. Then a description of the next comercial initiatives and OEMs in automation driving is shown followed by the final conclusions.



\subsection{Future challenges}

As shown in sections \ref{sec:02-sensors} and \ref{sec:03-problemsapplications} there exist many works that solve the most important perception competences in different ways, using different types of sensors and with a large variety of algorithms. However, there still exist different challenges that need to be solved in order to achieve a functional secure autonomous vehicle. 

\subsubsection{Technical challenges}


\emph{Plain 360 degree coverage is not enough}

Sensor setups in Automated Driving are usually focused on the areas relevant for the usual driving task: long range ahead from the vehicle, also long but no so much behind, and short-mid range in the laterals. This covers all the behavioral compentences enumerated in section \ref{sec:03-problemsapplications}, but there are some specific challenges like critical distances (too short and too large), oclusions, bad weather conditions and very fast events that are not usually covered and don't have a proper solution yet.


\begin{itemize}
    \item \emph{Very short distance} (including close to or below the 
    car)
        A person, animal or object right below the vehicle or intersecting 
        the path of the wheels represents a safety issue. While most situations
        can be anticipated when the element approaches the vehicle from the
        distance, it is not the case right before starting the vehicle, 
        while executing high accuracy maneuvers in certain conditions 
        (close to people or other moving elements).  
        
        This is a problem that can be tackled by adding redundant sensors in 
        specific positions. Some commercial 360-degree-view parking systems 
        \cite{gandhi2006vehicle} already seem to have a good visibility of vehicle close 
        surroudings. Waymo claims to have a special LiDAR monitoring this area
        and even below the vehicle. 
        
        In the future there will be a need of devices specific for this task 
        that can make automated vehicles even safer than human drivers in such 
        situations.
          
    \item \emph{Very long distance} 
        Detection and classification at 200 meters is an open issue.
        The resolution needed to monitor the whole area with the required 
        accuracy is overwhelming. Among current approaches, Ambarella 
        integrates a Ultra High Resolution camera (4k video, probably 8k) 
        (cited in \ref{sec:04-relevantdemos}), that is claimed to be enough for 
        discerning small objects at that target distance. The biggest problem
        is the raw computational power required to process so much information.
        
        A different approach could consist on sensors able to determine 
        saliency (a common term in artificial vision 
        \cite{Zhang2016a,Palazzi2018,Duthon2016}), 
        so that other adaptive sensors can focus on that area, increasing
        resolution, frame rate or accuracy. 
        This reminds in some way the saccadic vision system used in Dickmann's 
        pioneer vehicles \cite{Dickmanns1987,Gregor2002}, where a rotating 
        platform where used to take images of areas of interest.
        
        While the last part is something feasible today, for example using solid
        state LiDARs capable of random or adaptive sampling, the real challenge
        resides in the saliency part: design a sensor that can determine that 
        something very far away can be relevant, without falling in the
        brute-force approach.
        
    \item \emph{Indirect observations} (e.g. after oclusions)
        Humans are able to infer knowledge about the world based on subtle 
        hints, e.g. see that someone is behind a car from a glimpse through
        vehicle glasses, or attending at the shadow projected in the road. 
        
        There are two different ways to approach this problem:
        the first one is to process the observations of high resolution sensors 
        using and powerful processing algorithms. 
        The second way is to create 
        sensing technologies with super-human perception capabilities that can 
        directly register such events or elements. 
        Radars represent at present time the best example, since the propagation
        of electromagnetic waves has properties very different to usual 
        biological sensing capabilities. For example, automotive radars can 
        sometimes detect several vehicles in a row, totally occluded by the 
        first one. Another example is WaveSense's ground-penetrating radar 
        \ref{https://www.bloomberg.com/news/articles/2018-09-17/self-driving-cars-still-can-t-handle-bad-weather}
        previuosly presented in radar emerging technologies (section 
        \ref{sec:02-sensors}). The challenge remains open.
    
    \item {Environmental and weather conditions}    
        Section \ref{sec:02-sensors} summarizes the suitability of common 
        technologies under different conditions, some of which surpass human 
        capacities. 
        However, this is always an active field of research because perception 
        can solve what humans compensate with reasoning. Following the road 
        when most marks are covered by snow, or improving detection under heavy 
        rain are examples of problems that can be solved at sensing level
        without requiring further efforts on processing algorithms.
    
    \item \emph{Adaptation to very fast events} ???
    
\end{itemize}


\subsubsection{Implantation challenges}

The final purpose of research in autonomous driving is to create an autonomous vehicle that will be in the market (either for private custumers or for commercial use in a fleet). This means that the final product have to fulfill certain scalability, costs, and durability characteristics so its production is feasible. The final cost has to be as low as possible, and the durability of the vehicle need to be similar to the current vehicle's. The sensors are one of the most expensive parts and one of the most fragile, so their correct implantation is a key factor in the development of autonomous driving vehicles.

\emph{Production scalability and costs}

Mature technologies as visible light cameras and radars have already scaled up 
their production and reduced costs so that every vehicle can equip them without
a significant impact on its price. But it remains a challenge for LiDAR devices
and other breakthrough technologies.

The cost that is considered acceptable for a production vehicle, however, 
varies depending on the scenario. For private owned vehicle, it must be kept at
a rather small fraction of vehicle cost. In the case of fleet vehicles with a 
commercial use it can be higher because an Automated Driving system can 
compensate the cost of an employee.
It is difficult to provide estimations, because Automated Driving can have a 
significant effect in mobility, economiy an other factors. For a discussion
on costs and impact of Automated Mobility services, see \cite{Bosch2018}.

\emph{Durability and tolerance to failure}

The perception system of an Automated Vehicle must work flawlessly for long
periods under harsh conditions, as the rest of vehicle critical components. 
In case of failure, redundancy and emergency fallback routines must be able to 
mitigate the problem and drive the vehicle to a safe state, but it is a 
threat that has to be avoided.

Mechanical LiDARs have been around for about a decade, and are highly 
specialized devices mostly used with research purposes.
The controversial CEO of Quanergy, Eldada, claims 
(https://www.freightwaves.com/news/autonomous-trucking/quanergyceoripsvelodyne)
that mechanical LiDAR sensors are unsuitable for commercial 
automotive applications because the mean-time-to-failure (MTF)
"between 1,000 to 3,000 hours of operation" on the rotating 
components is far too low for industry requirements. Automakers want an MTF
of at least 13,000 hours.

Solid state lidars based on vibrating micro-mirror (MEMS) can reduce costs and increase laser resolution but 
still have movile parts (micro-mirror), which makes them susceptible to vehicle vibrations and more fragile.

External factors can affect sensors. A stone chip can crack a glass while 
driving at high speeds ways, and this is something that can affect the 
performance of video cameras and possibly LiDARs even when protected behind
a plastic or glass layer. It is desirable to create or improve sensing 
technology that can minimize the impact of that kind of events.

Another kind of external factors are intentional attacks. A radar can be jammed,
and a camera or a LiDAR can be blinded by the appropriate source of light.
Future sensors will have to be robust against external interferences.

\subsection{Commercial initiatives}

At the end of the pervious decade (2009), 
the most requested  Advanced Driving Assistance Systems (ADAS) 
\cite{Frost&Sullivan2010} were the Anti-lock braking system and the Parking 
Assistance by Warning. These systems cannot be classified
higher than SAE Level 0.

In the last decade the automotive market have grown the offer and complexity
of ADAS \cite{Perez2016}. The most advanced cars today equip an ensemble of 
ADAS that place them somewhere between SAE Levels 2 and 3 in the scale of Automated
Driving. Some examples are Tesla AutoPilot and Audi JamAssist, able to
drive the vehicle under user supervision in specific scenarios.

The fact that manufacturers are starting to talk about SAE Levels
is a sign that ADAS are completely integrated into the market and, thus, are
not a subject of research per se anymore.  


%\emph{RETOS FUTUROS??}

%\begin{figure}[h]
%    \centering
%    \includegraphics[width=1.00\textwidth]{"img/Resumen de algunas 
%aplicaciones 
%        y ADAS en el mercado_v3_trim"}
%    \caption{ADAS and AD applications in the market}
%    \label{fig:adas-apps}
%\end{figure}
%
%[3] 
%https://www.continental-automotive.com/en-gl/Landing-Pages/Industrial-Sensors/Products/ARS-404-21
%
%[4] 
%https://www.bosch-engineering.de/en/de/einsatzgebiete/schienenfahrzeuge/kollisionswarnung/kollisionswarnung\_1.html
%
%[5] https://www.mobileye.com/our-technology/
%
%[6] 
%https://brigade-electronics.com/products/radar-obstacle-detection/?gclid=EAIaIQobChMI2NTc\_ZKI3AIV6rXtCh1akgc6EAAYASAAEgK5xfD\_BwE
%
%[7] 
%https://www.bosch-mobility-solutions.com/en/highlights/automated-mobility/driver-assistance-systems-for-urban-areas/
%
%[8] https://www.nissan-global.com/EN/TECHNOLOGY/OVERVIEW/emergency\_brake.html
%
%[9] 
%https://www.zf.com/corporate/en\_de/products/product\_range/cars/cars\_automatic\_emergency\_braking\_function.shtml
%
%[10] https://www.mobileye.com/our-technology/
%
%[11] 
%https://support.volvocars.com/en-CA/cars/Pages/owners-manual.aspx?mc=y286\&my=2016\&sw=15w17\&article=d3d274ecedf1c586c0a801e8004927e7
%
%[20] https://www.valeo.com/en/lane-change-assistance-system/
%
%[30] 
%https://www.audi-mediacenter.com/en/techday-piloted-driving-the-traffic-jam-pilot-in-the-new-audi-a8-9276/automated-driving-at-a-new-level-the-audi-ai-traffic-jam-pilot-9283


\subsubsection{OEMs in Automated Driving}
\label{sec:oem-ad}

Back in 2010, most traditional vehicle manufacturers did not consider Automated
Driving as a priority. In the last years, the achievements of technological 
pioneers (Google/Waymo, Uber, Tesla among others) gave place to early alliances
between those companies that had the technology with OEMs that had the
platform, the experience and the market.

By the end of 2018, all the important brands are involved in a race for creating
the first highly automated vehicle (SAE Levels 4 and 5). It is difficult to
get information about their research further than public demonstrations and
marketing products. However, their alliances with other companies, startups
and technology/research centers are easier to trace and can hint about their 
approach to Automated Driving.


Figure \ref{fig:oem-ad} gathers the most important coallitions for Automated
Driving with OEMs involved.



\begin{figure*}[p]
    \centering
    \includegraphics[width=1.00\textwidth]{"img/OEM_jp_trim"}
    \caption{OEM projects and alliances in Automated Driving}
    \label{fig:oem-ad}
\end{figure*}

%\begin{table}
%    \caption{OEMs with Automated Driving projects}
%    \label{tab:oems}
%    \begin{tabular}{ | l | l | l | l | l | l | l | }
%    \textbf{OEMs} & \textbf{Test side} & \textbf{Technologies} & \textbf{Since}
%     & \textbf{Collaborations} & \textbf{Forecast} & \textbf{Test fleet} 
%    \\
%    \hline
%    Ford & Detroit, Arizona \& California (U.S.A.) & AI, LiDAR, and mapping  & 
%    \~2016 & Argo, Velodyne, SAIPS, civilmaps. & Level 4 (2021) & Fusion 
%Hybrid 
%    sedans \~100 by 2018 
%    \\
%    GM & Detroit San Francisco \& Scottsdale, Arizona (U.S.A.) & Lidar, very 
%    accurance map, radar, camera & \~2016 & Google’s Waymo and Jaguar-Land 
%    Rover & before 2020 (Fortune) & Fifty vehicles have been built by GM 
%(2017) 
%    \\
%    Renault-Nissan & Japan, EE.UU. \& China & Maintains speed, Steering 
%    control, Front radar, Lidars & \~2017 & Transdev, Microsoft and TechCrunch 
%    (from Oath)  & Fully autonomous car within the next 10 years. Level 3 -> 
%    2020 & --- 
%    \\
%    Daimler & Germany & Vision, data fusion, radar. & 2015 (Truck \& F015) & 
%    Bosch & 2020 & Commercial cars (level 2) 
%    \\
%    Volkswagen Group (Audi) & Germany & Lidar, data fusion, adaptive cruise 
%    control, self-parking \& TJA  & 2015 & Audi -> Delphi (2015);  Aurora 
%    (2017) & 2025 (level 4) & Commercial cars (level 3 -> Traffic Jams) 
%    \\ 
%    BMW & Germany, China & Vision, lidar, DGPS & 2011 & Intel,  and With Baidu 
%    \& Nokia’s HERE   & Level 5 autonomous car on the road by 2022. & 
%    Commercial cars (level 2) 
%    \\
%    Waymo & California (U.S.A.) & Lidar, vision system, radar, data fusion, RT 
%    Path plan.. & 2010 & Fiat-Chrysler,  Velodyne. & --- & 100 autonomous 
%    Pacifica minivans  
%    \\
%    Volvo & Sweden. \& Uber: San Francisco, Pittsburgh  & Vision, lidar, GPS, 
%    V2I & 2011 & Uber (U.S), Autoliv (Sweden)  & \~2020 & Commercial cars 
%    (level 2) 
%    \\
%    Tesla & U.S.A. & Camera, radar, AI & \~2015 & Apple, Mobileye and Nvidia & 
%    \~Full automated 2020  & Commercial cars (level 2) \\
%    Hyundai & South Korea & AI, LiDAR, Camera & 2014 & KIA, Aurora & AD Level 
%    3-> Highways by 2020 and to city streets by 2030 & --- 
%    \\
%    \end{tabular}
%\end{table}

\subsection{Conclusions}

Here it has been presented a survey about one of the most critical parts in autonomous vehicles, their sensors. Choosing the sensors configuration of an autonomous vehicle can be challenging. For all the perception competences, each sensor has different strengths and weakness. The task in which one sensor is not good, the other outperforms and vice versa. Usually the best solution consists in getting information from more than one type of sensor and fuse their information. It creates a more robust  perception system as it has more data variety from different sources and also increases the safety as the fault of an specific sensor can be managed by another one. As all of these advantages, it also presents some challenges like finding a proper way to calibrate all those sensors, or making good decisions when two sensors have different outputs, or in other words, make a proper fusion of all the information. Computational power and energy consumption are also related with the sensors choice. The more data the vehicle gets, the more computational power it will be needed with it corresponding energy consumption. That means that, if the vehicle has more sensors, the final cost will increase and therefore it will be more difficult  to produce those vehicles.

This survey has reviewed all the different sensors technologies avalible, describing their characteristics and how are they applied to get information usefull to solve the main perception competences. It gives a global vision of different sensor configurations and techniques to obtain usefull information from the environment.

The relevant works and demos provide a good perspective of how different manufacturers and research groups do perception tasks and wich kind of sensors they use for that purpose.

Finally, the section \ref{sec:05-commercialsystems}, which talks about commercial sensor systems fo ADAS can form an intuition about where are the manufacturers going in the autonomous vehicle process and how are they planning to get there.



%As the technology advances, new sensors with new specifications and charasteristics appears in the market [research with new platforms]


% use section* for acknowledgment
\section*{Acknowledgment}

This work has been partly supported by ECSEL Project ENABLE-S3 
(with grant agreement number 692455-2), by the Spanish Government through
CICYT projects (TRA2015-63708-R and TRA2016-78886-C3-1-R), and Comunidad de
Madrid through SEGVAUTO-TRIES (S2013/MIT-2713).


% Can use something like this to put references on a page
% by themselves when using endfloat and the captionsoff option.%
%\ifCLASSOPTIONcaptionsoff
%  \newpage
%\fi



% trigger a \newpage just before the given reference
% number - used to balance the columns on the last page
% adjust value as needed - may need to be readjusted if
% the document is modified later
%\IEEEtriggeratref{8}
% The "triggered" command can be changed if desired:
%\IEEEtriggercmd{\enlargethispage{-5in}}

% references section

% can use a bibliography generated by BibTeX as a .bbl file
% BibTeX documentation can be easily obtained at:
% http://mirror.ctan.org/biblio/bibtex/contrib/doc/
% The IEEEtran BibTeX style support page is at:
% http://www.michaelshell.org/tex/ieeetran/bibtex/
%\bibliographystyle{IEEEtran}
% argument is your BibTeX string definitions and bibliography database(s)
%\bibliography{IEEEabrv,../bib/paper}
%
% <OR> manually copy in the resultant .bbl file
% set second argument of \begin to the number of references
% (used to reserve space for the reference number labels box)
%\begin{thebibliography}{1}

%\newpage

\bibliographystyle{IEEEtran}
\bibliography{IEEEabrv,Tecnalia_AD-2018_review_sensors_perception_ADAS}

%\end{thebibliography}

% biography section
% 
% If you have an EPS/PDF photo (graphicx package needed) extra braces are
% needed around the contents of the optional argument to biography to prevent
% the LaTeX parser from getting confused when it sees the complicated
% \includegraphics command within an optional argument. (You could create
% your own custom macro containing the \includegraphics command to make things
% simpler here.)
%\begin{IEEEbiography}[{\includegraphics[width=1in,height=1.25in,clip,keepaspectratio]{mshell}}]{Michael Shell}
% or if you just want to reserve a space for a photo:


\begin{IEEEbiography}[{\includegraphics[width=1in,height=1.25in,clip,keepaspectratio]{img/foto_kike}}]
    {Enrique Mart\'i} is a Senior Researcher on Automated Driving at Tecnalia 
    Research and Innovation since 2017. He holds a B.E. degree in Computer
    Sciences (2009), a M.Sc. in Computer Science and Technology (2010) and a 
    PhD (2015) from University Carlos III de Madrid, with a thesis on sensor
    fusion systems sensitive to context information. 
    He has more than 10 years of experience in Sensor Fusion and Estimation,
    including participation in national and European R\&D projects and creation
    of commercial information fusion products for large scale surveillance.
    He has worked as Lead Engineer in Navigation and Localization systems in
    a private company building custom UAVs for industrial inspection.
    His research interests include sensor technologies, information fusion,
    estimation techniques, machine learning and optimization applied to
    automated vehicles.
\end{IEEEbiography}

% if you will not have a photo at all:
\begin{IEEEbiography}[{\includegraphics[width=1in,height=1.25in,clip,keepaspectratio]{img/Foto_miguel}}]
{Miguel \'Angel de Miguel}
    was born in Madrid, Spain in 1993. He received the B.S degree in 
    electronics engineering in 2015 and the M.S. degree in industrial
    engineering in 2017, both from University Carlos III de Madrid.
    In 2013 he joined the Intelligent Systems Lab where he has collaborated
    in industrial research projects for three years. Currently, he is a Ph.D. 
    student and an assistant lecturer in the university he studied.
    His research interests include the areas of path planning and control 
    with a focus on applications for autonomous vehicles.
\end{IEEEbiography}


% insert where needed to balance the two columns on the last page with
% biographies
%\newpage
\begin{IEEEbiography}[{\includegraphics[width=1in,height=1.25in,clip,keepaspectratio]
        {img/foto_fernando}}]
    {Fernando Garc\'ia} is Visiting Professor at Universidad Carlos III de
    Madrid. Where he focuses his researches in Intelligent Vehicles and
    Intelligent Transportation Systems involving the use of Computer Vision, 
    Sensor Fusion, and Human Factors, as well as Vehicle Communication. 
    He has been recipient of the 
Master in Robotics and Automatics Scholarship at Universidad Carlos III de 
Madrid in 2008, the Barreiros Foundation Award in the domain of Automotive and 
Vehicle Applications in Spain in 2014 and finalist to the best PhD Thesis 
Dissertation Award in the period 2013-2015 given by ITSS Spanish Chapter, and 
the award to the innovation in the automotive sector, diagnosis "Galeria de 
innovación Motortec 2015".  He has organization experience in the ITSS 
conferences  and is member of the BoG and Vice-president of the Spanish Chapter 
of the IEEE-ITSS since January 2017.
\end{IEEEbiography}

\begin{IEEEbiography}[{\includegraphics[width=1in,height=1.25in,clip,keepaspectratio]{img/FotoPaperJosh}}]
    {Joshu\'e P\'erez} is a Research leader on Automated Driving at Tecnalia 
    Research and Innovation, since 2015. He received the B.E. degree in 
    electronic engineering from the Sim\'on Bol\'ivar University, Venezuela, in 
    2007. His M.E. degree and his Ph.D. degree from the University Complutense 
    of Madrid were achieved in 2009 and 2012, respectively. He received the 
    best PhD thesis award in intelligent control from the Spanish Committee of 
    Automation (CEA). He has more than 11 years of experience in the 
    Intelligent Transportation System field, where he has been working in 
    national and European projects ranging from ADAS up to fully automated 
    vehicles. His research interests include fuzzy logic, modeling, control, 
    path planning, Cybercars, cooperative maneuvers, automated vehicles, Shared 
    Control, and vehicle standards. He is associate editor of IEEE Intelligent 
    Vehicle Symposium. In 2014, he was guest editors of the IEEE Intelligent 
    Transportation Systems Magazine, in the special issue of ELECTRO-MOBILITY. 
    He has technically led 20+ EU and national projects for demonstrating 
    different ADAS technologies. He has more than 100 publications related to 
    Automated Driving, ADAS and cooperative maneuvers.
\end{IEEEbiography}

% You can push biographies down or up by placing
% a \vfill before or after them. The appropriate
% use of \vfill depends on what kind of text is
% on the last page and whether or not the columns
% are being equalized.

%\vfill

% Can be used to pull up biographies so that the bottom of the last one
% is flush with the other column.
%\enlargethispage{-5in}



% that's all folks
\end{document}


