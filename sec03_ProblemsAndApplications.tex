

En esta sección hablaremos de qué problemas hay que resolver en conducción automatizada y cómo se categorizan (SAE Levels, behavioral competencies NHTSA / California PATH, \url{https://storage.googleapis.com/sdc-prod/v1/safety-report/waymo-safety-report-2017-10.pdf}).

Desarrollar el "arco argumental": problema -> descomposición en habilidades/partes -> información necesaria.

Problems of interest in ADAS at different automation levels.
Source: \href{http://www.mobileye.com/wp-content/uploads/2013/09/AEB_test_procedures_aug12.pdf}{MobilEye slides}

\subsection{SAE Automation Levels}

\textbf{Level 0} comprises traditional assistance mechanisms that can be detected monitoring mechanical parts of the vehicle
\begin{itemize}[leftmargin=20mm,labelsep=5.8mm]
    \item	\gls{abs}
\end{itemize}

\textbf{Level 1}
\begin{itemize}[leftmargin=20mm,labelsep=5.8mm]
    \item	Cruise control
    \item	\gls{aeb} Tightly related with \gls{fcw}. Three different cases
\end{itemize}

\textbf{Level 2}
\begin{itemize}[leftmargin=20mm,labelsep=5.8mm]
    \item   \gls{acc}
\end{itemize}

\textbf{Level 3}
\begin{itemize}[leftmargin=20mm,labelsep=5.8mm]
    \item Control sharing techniques (¿Sección propia?)
\end{itemize}

\textbf{Level 4}
\begin{itemize}[leftmargin=20mm,labelsep=5.8mm]
    \item	...
\end{itemize}

\textbf{Level 5}
\begin{itemize}[leftmargin=20mm,labelsep=5.8mm]
    \item	...
\end{itemize}

\subsection{Behavioral competencies}


\begin{table}[H]
    \caption{Behavioral competences}
    %\centering
    %% \tablesize{} %% You can specify the fontsize here, e.g.  \tablesize{\footnotesize}. If commented out \small will be used.
    \begin{tabularx}{\linewidth}{LlL}
        \toprule
        \textbf{Competence}	& \textbf{\#} & \textbf{Behaviour}	\\
        \midrule
        \multirow{4}{*}{Automatic Traffic Sign Detection and Recognition (TSDR)}
         & 8    & Detect Speed Limit Changes, Speed Advisories, Traffic Signals 
         and Stop/Yield Signs \\
         & 8    & Detect Access Restrictions (One-Way, No Turn, Ramps, etc.) \\
         & 8    & Detect Temporary Traffic Control Devices \\
         & 6, 8 & Detect Passing and No Passing Zones  \\
         \midrule
         \multirow{4}{*}{Perception of the environment}
         & 8 & Detect Lines \\
         & 6, 8 & Detect Detours  \\
         & 6 & Detect faded or missing moadway markings or signage and/or other 
         temporary changes in traffic patterns \\
         & 9 & Perception in unanticipated weather or lighting conditions outside of 
         vehicle’s capability (e.g. rainstorm) \\
         \midrule
         \multirow{2}{*}{Vehicles as obstacles detection}
         & 10, 12, ¿13? & Detect Non-Collision Safety Situations (e.g. vehicle 
         doors ajar) \\
         & 10, 11, 12, 13 & Detect Stopped Vehicles, Emergency Vehicles, Lead 
         Vehicle, Motorcyclists, School Buses \\
         \midrule
         \multirow{3}{*}{Perception of vehicles behaviour}
         & 6, 11(, 8) & Detect Lane Changes, a Merging Vehicle \\
         & 6, 11 & Detect Vehicles Parking in the Roadway and Locate Spaces to 
         parky \\
         & 6, 8 & Detect Encroaching Oncoming Vehicles \\
         \midrule
         \multirow{4}{*}{Other obstacles detection}
         & 6(, 1)  & Detect Static Obstacles in the Path of the Vehicle \\
         & 6, 8, 9, 10, 11, 12 & Detect Pedestrians and Bicyclists at 
         Intersections, Crosswalks and in Road (Not Walking Through 
         Intersection or Crosswalk) \\
         & 10, 11, 12 & Detect Animals \\
         & 10, 12, 13 & Detect instructions from Work Zones and People 
         Directing Traffic in Unplanned or Planned Events, Police/First 
         Responder Controlling Traffic, Construction Zone Workers Controlling, 
         Citizens Directing Traffic After a Crash (Overriding or Acting as 
         Traffic Control Device) \\
         \midrule
         \multirow{1}{*}{Others}
         & 2 & Detect Conditions Involving Vehicle, System, or Component-Level 
         Failures or Faults (e.g. power failure, sensing failure,sensing 
         obstruction, computing failure, fault handling or response)  \\

        \bottomrule
    \end{tabularx}
\end{table}

\subsubsection{Automatic Traffic Sign Detection and Recognition (TSDR):} %[Definition? ]. 
There are two main tasks for getting information from on road traffic signs: Traffic sign detection (TSD) which consists on finding the location, orientation and size of traffic signs in natural scene images, and Traffic Sign Recognition (TDR), classifying the detected traffic signs into their specific sub-classes in order to extract the inforation that they are providing to drivers.
It also has two different applications: Real time detection and recognition, used in ADAS or autonomous driving, and an automatic road traffic sign mapping system, used for generating a database of all the trafic signs of an area. This last application doesn't need to work on real-time. 
The two main sensors used for these tasks are monocular cameras in different configurations (a single one, multiple focals or multiple cameras), and LiDAR sensors.
Below are shown the most relevant solutions acording to the type of sensor, and the technlogy used.

\begin{flushleft} \textbf{Camera based solutions:}
Cameras are the most common sensor for TSDR. They can be used for TSR, TSD or both at the same time.
As an example of TSR, \cite{frejlichowski2015application} proposes a method based on the Polar-Fourier Greyscale Descriptor, which applies the information about silhouette and intensity of an object. \cite{gao2015learning} uses a learning method based on an histogram intersection kernel to quantizize features, and look-up table approach to encode them.
For TSD, \cite{zhang2017real} proposes a method based on a fast Convolutional Neural Network (CNN) inspired in the YOLOv2 network. With this algorithm it is possible to detect the position of the traffic sign and classify it into Mandatory (blue ones) Danger (Trinagle) and Prohibitory (red circle). \cite{villalon2017traffic} detects stop and yield signs with a statistical template built using color information in different color spaces (YCbCR and ErEgEb). TSD techniques can also be applied to traffic light detection, as in \cite{hosseinyalamdary2017bayesian}, where a Bayesian inference framework to detect and map traffic lights is described. A different approach is proposed by \cite{gu2011traffic} that uses a dual focal camera system composed of a wide angle camera assisted with a telephoto camera which is moved by a mirrors system in order to get higher quality images of the traffic signs.
Camera sensors can also perform both tasks, detection and recognition as is shown in the following works. \cite{miyata2017automatic} uses local binary pattern methodd for detecting speed signals and a neural network for the recognition of the numbers of the sped limit sign. \cite{yang2016towards} presents a fast detection method based on traffic sign proposal extraction and classification built upon a color probability model and a color HOG combined with a convolutional neural network to further classify the detected signs into their subclasses.
\cite{wali2015automatic} performs detection and recognition tasks using a RGB colour segmentation and shape matching followed by support vector machine (SVM) classifier. \cite{timofte2014multi} works with an eight roof-mounted cameras configuration that takes images every meter and are processed offline combining 2D and 3D techniques to create an available database, with more than 13,000 traffic signs annotations.
\end{flushleft}

\begin{flushleft} \textbf{LiDAR based solutions:}
LiDAR sensors are also used, mainly for TSD, as they provide 3D information, used to localize the position of the sign and its shape, and information about the reflectiveness of the surface, which helps detecting traffic signs as they usually are high reflective. \cite{gargoum2017automated} performs detection in three steps: first the pointcloud is filtered by intensity of the laser reflected, then a clustering is apllied to detect possible candidates and after that, a further filter based on the lateral position, elevation and geometry is applied to extract the sign. \cite{weng2016road} goes one step further and is able to also make a primary classification attending to the sign shape (rectangular triangular and circular).
\end{flushleft}

\begin{flushleft} \textbf{Sensors Fusion solutions:}
A system that combines both sensors, LiDAR and Cameras can improve the sign detection and recognition as it has the advantages and the information of both sources. \cite{zhou2014lidar} trains a SVM with 10 variables: 9 of different color spaces provided by the camera (RGB, HSV, CIEL*a*b*) and the intensity provided by the LiDAR. After 3D geometric feature verification of the detected signs, the classification is made using HOG features and a linear SVM. \cite{guan2018robust} method, first detects traffic signs from mobile LiDAR point clouds with regard to a prior knowledge of road width, pole height, reflectance, geometrical structure, and traffic-sign size, then traffic signs are normallized to finally perform classification based on a supervised Gaussian–Bernoulli deep Boltzmann machine model.
\end{flushleft}

\newpage
\begin{flushleft}
\textbf{Summary of sensors and their use for TSDR:}
\begin{itemize}[leftmargin=20mm,labelsep=5.8mm]
\item \textbf{LiDAR:} Filtering for intensity (traffic signal has high Reflectance), 3D position and shape. Clustering for sign candidates
\item \textbf{Camera}
\subitem \textbf{Monocular B/W:} Feature extraction.
\subitem \textbf{Monocular Color:} Color information. Classification with different methods:  SVM, CNN, ..., or both, detection and classification.
\subitem \textbf{Multiple cameras:} Mapping signals in roads, 3d position.
\subitem \textbf{Multiple focals:} Short focal for detection and long focal for higher quality sign image.
\item \textbf{Fusion LiDAR and Camera:} Using LiDAR intensity and 3d position and shape information for detecting, and camera colour and shape information for recognising
\end{itemize}
\end{flushleft}

\subsubsection{Perception of the environment}
The purpose of this compentence is to understand the environment of the vehicle: the road. It is usually divided into 2 tasks.
The first one is to detect and recognise the road marks that complemens the traffic signs (Stop turn, stopping lines) and the lane marks that delimit those lanes. With this information a map can be generated using localization information provided by a high precision GNSS based sensor in that places all the detected marks in their corresponding global position.
The second task is related to the localization of the vehicles in the road. It can be either on map generated by the first map, or in unknown roads. 
There exist two critical facts that need to be considered in this competence: As most of the perception algorithms, this one requires a very high reliability, which is hard to get. The information provided by this kind of algorithms can be used to center the vehicle in the lane (lane-level localization), so a bad detection can be dangerous. It also has to work in very different kinds of environments, not only due to the different lanes signals that could vary from one country to another, but for different weather conditions. Different light intensity (Sunny day, night, shadows), different visibility conditions (fog snow rain) and different reflectivity of the road (wet or dry asphalt). These conditions can make hard to detect correctly the road marks.
In the following lines, the most relevant works about this topic are presented, orginized by the type of sensor they use.

%Semantic representation of the lanes
%Using the reflectivity information recorded by lidars enable the our system to detect lane markers in the presence of shadows, against direct sunlight and even at night

\textbf{Camera based solutions:}
Road infrastructures are designed so that human drivers can see them, they are optimized to provide visual information in different weather conditions (reflective painting, high contrast colours, etc). As cameras are also able to obtain visual information, they are the most commonly used sensor to do this task. There are different configurations used in the literature, that can be classify into:
\underline{Single Monocular}: It is the simplest configuration. Using only one camera looking at the road in front of the vehicle it is possible to estimate the road shape and its lanes, the position of the vehicle in the road and to detect roadmarks  that provide driving information. A survey of the most relevant algorithms used for this purpose, mainly for camera sensors is presented in \cite{hillel2014recent}.
\underline{Multiple Monocular cameras}: Some works use the around view monitor (AVM) technology. This configuration uses multiple cameras placed around the vehicle (tipically 4, one at each side of the vehicle) in order to get 360º visual information of the surroundings of the vehicle. \cite{lee2017avm, kum2013lane}. A different configuration is used in \cite{Ieng2003}, where two lateral cameras are used to localize the vehicle. 
\underline{Binocular or Stereo}: The main advantage of using binocular cameras is that, as they provide 3D information, the groundplane can be detected easily. With this onformation of the groundplane, the road boundaries and the roadmarks detection is improved. \cite{schreiber2013laneloc, ozgunalp2017multiple}

\textbf{LiDAR based solutions:}
LiDARs main ussage is to detect the ground plane of the road and the limits of the road, and detect obstacles that could oclude parts of the road.
In recent works, LiDAR based solutions also takes advantage of the higher reflectivity of lane marks (they are painted with  white  high reflective paint) with respect to road (which is black) to detect them. LiDARs, as they are active sensors, are less perturbable by light conditions (the can work even at night) that cameras, but the resolution and the contrast of thesedetections are lower.
As some roads are not good maintained, this higher reflectivity is lost in some places, so LiDAR information is usually fused with cameras that are able to percive non reflective but white lane marks as they have higher contrast.
Some works use 2D LiDAR sensor to extract the road geometry and roadmarks. It is a cheaper sensor, but it provides less information as it only measure distances in one plane. \cite{nie2012camera, kim2015lane, lee2017avm}
Meanwhile, 3D LiDAR gives a more dense pointcloud with more information, improving lane and roadmarks detection, but with a higher cost. A curve detector and pavement markings detector are presented here \cite{yang2012automated, zhao2012curb, li2013new, Zhang2016}.
%In some works LiDARs are used to detect road boundaries and map them into images in order to get training data for segmentation algorithms Used to classify images. \cite{}

\textbf{Radar based solutions:}
The main advantage of Radar sensors is their ability to work in all weather conditions (darkness, rain, fog, snow, etc). But due to the poor image contrast of these types of sensors, the only usefull information for that competence that can be extracted is the position of the obstacles (in order to know oclusions of the road) and the road itself. Road, as it is plain, acts as a mirror for a radar sensor, returning very little power of the emited, the sides of the roads return a slightly higher amount of power, and obstacles return a high amount of power. So road limits can be estimated using radar images, getting a maximum error of half a lane at zero distance from the host vehicle and remains within a full lane width even at 50 meters distance from the host vehicle. This radar information the is fused with camera images in order to improve both detections \cite{kaliyaperumal2001algorithm, ma2000simultaneous, Janda2013}

\subsubsection{Vehicles as obstacles detection}



\subsubsection{Perception of vehicles behaviour}



\subsubsection{Other obstacles detection}

