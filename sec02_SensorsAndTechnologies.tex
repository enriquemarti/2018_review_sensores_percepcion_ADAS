
En esta sección se discutirán las principales tecnologías de sensores (opcional: revisión de tecnologías emergentes).

Debemos cubrir el arco "sensores -> datos -> información", que lleva desde los sensores crudos hasta información relevante en AD. Esto nos permite enganchar con la sección Problems and applications.

\begin{figure}[h]
    \centering
    \includegraphics[width=0.95\textwidth]{"img/data_vs_information"}
    \caption{Perception as a bridge between raw data and useful information}
    \label{fig:data-vs-information}
\end{figure}


\subsection{Relevant information domains}

The task of a perception system is to bridge the gap between sensors providing 
data, and decision algorithms requiring information.
A classical differentiation between both terms is the following: data is 
composed by raw, unorganized facts that need to be processed. 
Information is the name given to data that has been processed, organized, 
structured and presented in a proper context.

The fields of study of ADAS and Automated Driving systems are large and 
heterogeneous.
This makes necessary to use many types of information, that can be organized 
according to the following taxonomy / The following taxonomy gives an idea of 
the heterogeneity and complexity of ADAS/AD problems domain:

\begin{table}[H]
    \caption{Information taxonomy in Automated Driving domain}
    %\centering
    %% \tablesize{} %% You can specify the fontsize here, e.g.  \tablesize{\footnotesize}. If commented out \small will be used.
    \begin{tabularx}{\linewidth}{lrL}
        \toprule
        \textbf{Category} & \textbf{\#}	& \textbf{Information type}	\\
        \midrule
        \multirow{2}{*}{Ego-vehicle}
         & 1 & Kinematic/dynamic (includes position) \\
         & 2 & Proprioceptive (components health/status) \\
         \midrule
         \multirow{3}{*}{Passengers/driver}
         & 3 & Driver awareness/capacities \\
         & 4 & * Driver intentions (mind model)  \\
         & 5 & Passenger status (needs, risk factors) \\
         \midrule
         \multirow{4}{*}{Environment}
         & 6 & Spatial configuration: location, size, shape, fine features 
         \\
         & 7 & Identification: class, type, identity \\
         & 8 & Regulation and semantics: traffic signs, road marks, other 
         elements \\
         & 9 & Contextual factors: weather, driving situation(e.g. jam, 
         highway, off-road) \\
         \midrule
         \multirow{4}{*}{External actors}
         & 10 & Spatial features: location, size, shape, fine features  \\
         & 11 & Kinematic/dynamic: position, motion \\
         & 12 & Identification: class, type, identity \\ 
         & 13 & Semantic features: vehicle lights, pedestrian clothes, gestures 
         \\
         & 14 & * Situational engagement: collaborative/aware 
         (adult pedestrians, other vehicles) vs non-collaborative/unaware 
         (animals, children) \\ 
        \bottomrule
    \end{tabularx}
\end{table}

Elements with an asterisk are derived information. This is, something that can 
be inferred from sensed data but not directly observed. It is mostly related 
with internal state of external entities, as the intentions of human beings and 
animals.

In this survey we are focused in perception of external elements, that is, the
two last categories.

\subsection{Principal sensor technologies for perception}

There are a few categories of sensors that are included in almost every ADAS/AD
system and are sufficient to cover any reasonable perception need. 
These categories are: artificial vision, radar, Lidar and 
ultrasonic.

Most of them are ranging sensors: the measured magnitude is distance to 
external objects, by means of the round-trip delay time of a electromagnetic 
wave (radar), a pulse of structured light (Lidar) or sound (ultrasonic).
Artificial vision, in its most basic form, registers a grid of intensities and 
colors. This allows to extract aditional information about the world, including 
traffic elements that are encoded visually to allow being interpreted by humans 
(signals, lanes, traffic lights).

Sensors provide raw data that sometimes can be used directly as useful
information (e.g. GPS giving location/speed), or that must be processed/fused.

Tabla: qué sensores son útiles para obtener cada tipo de información.

\begin{table}[H]
    \caption{Sensor technologies and proposed uses}
    %\centering
    %% \tablesize{} %% You can specify the fontsize here, e.g.  \tablesize{\footnotesize}. If commented out \small will be used.
    \begin{tabularx}{\linewidth}{cLL}
        \toprule
        \textbf{Type}	& \textbf{Sensor type}	&  \textbf{Applications}\\
        \midrule
        \multirow{4}{*}{Artificial vision}	
        & Monocular single camera   & \gls{fcw} \cite{Dagan2004}, \gls{ldw}, \gls{acc} \cite{Stein2003}, road marks/signs detection, visual odometry for ego-motion \cite{Scaramuzza2008,Nguyen2013,Bak2012}, Road boundary detection for run-off road prevention based on the fusion of video and radar \cite{Janda2013} \\
        & Monocular multicamera     & \cite{Ieng2003} lateral cameras for lane detection fused with GPS for enhanced location \\
        & Stereo                    & - \\
        & Non-visible spectrum      & Pedestrian detection \cite{Besbes2015} \\
        & Time Gated CMOS           & see https://www.brightwayvision.com/technology/ \\
        \midrule
        \multirow{5}{*}{Ranging}
        & Radar                 & Ground+obstacle detection with mm-wave radar \cite{Reina2015}, Road boundary detection for run-off road prevention based on the fusion of video and radar \cite{Janda2013} \\
        & Lidar (single layer)  & - \\
        & Lidar (multilayer)    & Rapid Inspection of Pavement Markings Using Mobile Laser Scanning Point Clouds \cite{Zhang2016} \\
        & Ultrasounds           & Neighbor vehicle misbehavior \cite{Liu2015}, also for (taken from that work) ''vehicle tracking and classification [7,22–25], obstacle detection and mapmaking [26,27], vacant parking slot detection [28], smart traffic signaling [29], ultrasonic ranging and localization [30]'' \\
        & Structured light    & -  \\
        & Time-of-flight       & - \\
        & Time Gated CMOS     & (also in Artificial Vision) \\
%        \midrule
%        \multirow{3}{*}{Egomotion}
%        & GNSS      & Positioning, speed \\
%        & IMU       & Orientation (inclination, heading), change of speed \\	
%        & Odometer  & Speed \\
        
        \bottomrule
    \end{tabularx}
\end{table}

In \cite{Lundquist2011} we can find a nice review of sensor/algorithms for road 
related applications: lane detection, road geometry estimation, etc.
The work \cite{Yebes2015} talks about infering 2.5D or 3D features from monocular 
vision.
Pot-hole detection using monocular vision
Road condition with on-board sensors in \cite{CastilloAguilar2015} (tire slippage).

Monocular vision is the most popular technique due to the low cost of sensors and 
having a large research community with a high maturity degree.

%\begin{figure}[h]
%    \centering
%    
%\includegraphics[width=0.95\textwidth]{"img/imaging-technologies-for-automotive-2016-report-by-yole"}
%    \caption{Ranging technologies}
%    \label{fig:ranging-techs}
%\end{figure}

%\begin{table}[H]
%    \caption{Features of spatial sensor technologies 
%([D]etection/[I]dentification, [G]ood/[M]edium/[B]ad)}
%    \begin{tabularx}{\linewidth}{Lcccccccc}
%        \toprule
%        \textbf{Technology} & \textbf{veh} & \textbf{ped} & \textbf{signals} & 
%\textbf{road marks} & \textbf{other} & \textbf{dense} & \textbf{resol} & 
%\textbf{fog/rain/snow} \\
%        \midrule
%        Mono (visible) 		& D/I & D/I & D/I & D/I & D/I & G & G & M/B \\
%        Mono (near IR) 		& - & - & - & - & - & - & - & - \\	
%        Mono (far IR) 		& - & - & - & - & - & - & - & - \\	
%        Stereo (visible) 	& - & - & - & - & - & - & - & - \\	
%        Time-of-flight 		& - & - & - & - & - & - & - & - \\	
%        Time Gated 			& - & - & - & - & - & - & - & - \\	
%        Structured Light 	& - & - & - & - & - & - & - & - \\	
%        Radar 				& - & - & - & - & - & - & - & - \\	
%        Lidar 				& - & - & - & - & - & - & - & - \\	
%        Ultrasounds 		& - & - & - & - & - & - & - & - \\		
%        \bottomrule
%    \end{tabularx}
%\end{table}

\begin{figure}[h]
    \centering
    \includegraphics[width=0.8\textwidth]{"img/plot_range-fov"}
    \caption{Range-FoV for depth sensors (3D sensing tecnology)}
    \label{fig:range-fov}
\end{figure}

\subsection{Artificial Vision}
Artificial vision is a popular technology that has been used for decades in 
disciplines as mobile robotics, surveillance, industrial inspection. It acquires
information about objects in the real world by analyzing their images as 
captured by photo and video cameras. Cameras are devices that gather light using
a sensor composed by a grid of thousands or millions of individual detection 
elements. The amount of light captured by each element is translated more or 
less directly into the intensity of a pixel in the resulting image. 

This technology offers interesting features, as the low cost of sensors --only 
some types-- and providing range of information types including spatial
(shape, size, distances), dynamic (motion of objects by analyzing their 
displacement between consecutive frames) and semantic (shape analysis).

Artificial vision technology face several challenges, especially in 
applications like automated driving:

\begin{itemize}
    \item Varying light conditions: driving is not limited to daylight 
    conditions. It can also happen at night, indoors, or even worse,
    during dusk or dawn with the sun close to the horizon. Dark spots,
    shadows and other effects difficult the implementation of reliable 
    artificial visible algorithms.
    
    \item High dynamic range in the scene (some areas are very dark 
    and some others are strongly illuminated): while varying light conditions
    can be mitigated through dynamimc exposure mechanisms, HDR conditions
    are more difficult to deal with. Most sensor technologies have a 
    limited capacity of capturing both extremes simultaneously, so that 
    information is lost in one or the two sides. In automotive applications, 
    this can mean detecting lane lines at the cost of not seeing preceding
    vehicles or vicecersa.
    
    
    \item Objects moving at great speed: cameras need higher exposures time as
    illumination is weaker. Fast moving elements appear blurred, which can 
    affect later processes as border or feature detection. Also, if the sensor
    does not capture light in its full surface simultaneously (rolling shutter)
\end{itemize}

In \cite{Pueo2016} some of these problems are analyzed from the perspective of
recording scenes in sports.
In order to deal with these difficulties, different technologies and solutions 
have been proposed. 

\begin{itemize}
    \item Rolling vs. Global shutter sensors. In rolling
    shutter cameras, either sensor technology (rows activated sequentially) or
    mechanical elements (a physical shutter moving to expose and oclude the 
    sensor), the elements of the sensor capture light at differents time 
    intervals. This have negative effects as: fast moving objects appearing
    distorted, or flickering lights (fluorescents, LEDs) creating images with 
    some parts illuminated and others dark. Rolling shutter effects
    can be corrected by compensating scene motion vector, as proposed in 
    \cite{Chia-KaiLiang2008}\cite{Chun2008}.
    Global shutter cameras, on the other hand, have the ability to capture 
    light in all the elements of the sensor simultaneously.
    
    \item High Dynamic Range imaging (HDR): common sensors in photographic and 
    industrial cameras offer a dynamic range of 60-75 dB (10 to 12.5 EVs),
    that is not sufficient in mixed illumination environments as entering or 
    exiting tunnels. Sony launched its IMX390 automotive sensor with an
    extended 120 dB range (equivalent to 20 EVs) and 2k resolution. 
    An automotive grade sensor combining HDR capabilities and 
    Near Infra-Red light detection is analyzed in \cite{Maddalena2005}. 
    In \cite{Strobel2013} a sensor with 130 dB range (global shutter) and up
    to 170 dB (rolling shutter) is presented for industrial safety application. 
    
    \item Refresh rate:
    
    \item Captured spectrum: apart from visible light cameras, either in color
    or grayscale, automotive applications have used infrared sensors. We can 
    distinguish to families: far infrared cameras (wavelength 900-1400 nm), 
    also known as thermal cameras, detect the emissions of hot objects, 
    including living beings. Thermal cameras are effective for pedestrian and
    animal detection \cite{OMalley2008}\cite{Besbes2015} in the dark and 
    through dust and smoke. FLIR manufactured a system that was integrated
    in BMW's NightVision system since 2009 cameras.
\end{itemize} 

Event-based vision is a bioinspired technology developed by Zurich University
and the ETH Zurich. It approaches the problem of processing and tracking fast 
moving images in a completely different way: instead of capturing fixed frames
at discrete times, the elements of the sensor are triggered asynchronously and
indepently when a change on the intensity is detected. Each activation is
called an event, and can be seen as something similar to the output of a 
feature detection algorithm for artificial vision applications.
Events can be grouped in time windows for getting a frame-like image, but this
window can be dynamically adapted for attending different time scales.
Sensor time resolution goes down to microseconds, and \cite{Mueggler2014} shows 
integration windows of 1 ms (~1000 fps) in regular indoor lightning conditions
to be sufficient for high speed tracking, thanks also to the 120 dB dynamic 
range of the sensor.
An additional advantage is that the output can be used in raw form for 
applications as visual odometry \cite{Censi2014} and SLAM \cite{Vidal2017}
relieving the CPU of time consuming operations on raw images.
More recently, a steering wheel control for automated driving systems 
based on deep learning has been shown in \cite{Maqueda2018}.
 
 
 
 Talk about 3D technologies??? estéreo, ToF, structured light, 
 

\subsection{Radar}

%Radar technology use high frequency electromagnetic waves

Automotive radars were initially designed around the 24 GHz band. This band was temporarily open in EU to allow development, test and initial launch to market of
automotive radar devices, but needed to be changed because it interferes with some radio services (including radio astronomy stations).
In 2004, EU finally allocated a permanent 5 GHz wide band around 79 GHz. Short 
distance applications as blind spot detection, parking assistance, jam 
assistance and pre-crash measures use the upper part (77-81 GHz), since it 
offers a better resolution. Long distance applications as ACC use a radar 
signal around 76-78 GHz. However, manufacturers are implementing multifrequency 
chips that can switch between different functions dinamycally.
This information is resumed in figure \ref{fig:radar-freqband}

Radar technology is a good tool to achieve 360 degrees coverage. It provides 
(https://blog.nxp.com/automotive/radar-camera-and-lidar-for-autonomous-cars)
obstacle position (as distance-angle) and size.
Radar is a very mature technology that offer robust results. One of its strengths is
independence of light and weather conditions. Detections are equally good with snow, rain or fog.
However, it can fail to provide a reliable output under some conditions. For example, ACC in a long curve can detect vehicles that are in a different lane or miss the target due to narrow field of view. In these cases, sensor fusion with another source (e.g. video signal) can largely improve results.

Metal objects cause reflections that can produce a spurious detection. Something as small as a metal can in the road can in some cases generate this effect.

%NOTAS IMANOL:
%Atenuación a partir de 10GHz (lluvia, gases).
%Radar freq. más alta 
%  - más resolución espacial -> directamente relacionado con long. onda
%  - más resolución velocidad -> igual
%  
%Potencia de emisión
%  - No hay estudios concluyentes sobre efectos en salud en función de 
%potencia                     


%\begin{figure}[h]
%    \centering
%    \includegraphics[width=0.8\textwidth]{"img/infineon_radar_bands_chart"}
%    \caption{Operating frequency bands of automotive radars. Taken from 
%Lachner, R. (2009). Development Status of Next Generation Automotive Radar in 
%EU. Tokyo: ITS Forum. Available at 
%http://www.itsforum.gr.jp/Public/J3Schedule/P22/lachner090226.pdf}
%    \label{fig:radar-freqband}
%\end{figure}

\subsection{LiDAR}
LiDAR (Light Detection And Ranging) is an active ranging technology that calculates distance to objects by measuring round trip time of a laser light pulse.
Devices apt for robotic and automotive applications use a low power near-infrared laser (900-1050 nm) that is invisible, eye-safe and able to measure up to 100 m under direct sunlight.
LiDARs can be used for mapping environments. Tipically, a rotating mirror is used to change the direction of the laser pulse, reaching 360º horizontal coverage. Commercial solutions use an array of emitters to produce several vertical layers (between 4 and 128) that, combined with rotation, generate a point cloud that represents the the environment.

In the last years, a FLASH based technology (also named as solid state LiDAR) has arrived to commercial products. It offers many advantages when it comes to operating under strong vibrations and dynamics, apart from being potentially smaller, cheaper and faster.
However, the market does not offer products combining high resolution and a wide field of view, so mechanical devices are the only option for full 360-degree coverage and environment mapping.

Manufacturers: Velodyne, Sick, Continental, LeddarTech and Quanergy


%\begin{figure}[h]
%    \centering
%    
%\includegraphics[width=0.85\textwidth]{"img/Freescale_factTable_sensorApplication"}
%    \caption{Applications of 77 GHz radar to ADAS}
%    \label{fig:freescale}
%\end{figure}




\begin{figure}[h]
    \centering
    \includegraphics[width=0.7\textwidth]{"img/information_types_sensors"}
    \caption{Sensor adequacy for relevant types of information}
    \label{fig:information_vs_sensors}
\end{figure}
