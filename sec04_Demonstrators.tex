This section describes some of the most relevant technological demonstrations, competitions, challenges and commercial platforms related with automated driving, starting from pioneering works in late 1980s until present day. Figure \ref{fig:tech-demos} shows the timeline, with the focus on the sensors equipped by each platform.

The timeline allows to discern different stages ("ages") in the development of Automated Driving technology, and to identify trends and approaches from the perception point of view for Automated Vehicles.

\begin{figure}[p] %[h]
  %\centering
  \includegraphics[width=0.95\textheight,angle=90,keepaspectratio]{"img/AD_demos_Timeline"}
  \caption{Timeline of relevant AD demonstrators and its sensor setup for 
      Perception systems}
  \label{fig:tech-demos}
\end{figure}

\subsection{Pioneer works (1980-2000)}

Pioneer works in Automated Driving starts around mid-1980s. The Bundeswehr 
University of Munich developed automated vehicles based on visual guidance, as 
VaMoRs \cite{Dickmanns1987} (a large van) and VaMP \cite{Gregor2002} (a 
Mercedes 500 SEL). Perception was built over a saccadic vision 
system: cameras mounted on a rotating platform that allowed them to focus in
relevant elements. Sixty transputers executed an intelligent 4-D approach to
object tracking, delivering a huge amount of computational power according to
the standards at that moment.

In early 1990s INRIA creates the concept of a "Cybercar", a small automated 
electric vehicle for shared urban use \cite{Parent1993}. Project Praxitèle
\cite{Massot1999} (a mixed public and industry initiative) led to a fully
functional prototype. In 1997 the Cybercar debuts in Schippol airport to
transport passengers, between the terminal and long stay parking 
\cite{Ozguner2007}. The vision and LiDAR based vehicle lacked pedals and 
steering wheels and moved autonomously in a dedicated lane that included 
semaphores and some pedestrian crossings.

Also in 1997, the National Automated Highway System Consortium presented a 
demonstration of Automated Driving functionalities \cite{Thorpe1997}, intended 
to be a proof of technical feasibility. 
The demo showed road following functionality based on vision sensors, distance 
maintenance based on LiDAR, vehicle following based on Radar and other 
functionalities including cooperative maneuvers and mixed environments.

At that time, relevant functional demonstrators strongly relied on visual 
processing. The University of Parma started in 1996 a project with its vehicle 
ARGO \cite{Broggi1998}, equipped with two cameras that allowed road following,
platooning and obstacle avoidance.
%They detected lane lines and 
%vehicles using classical image processing techniques including preprocessing 
%steps, feature extraction, model fitting and spatio-temporal filtering.
It completed over 2000 km of autonomous driving in public roads, and the full 
experience was compiled in a book \cite{Broggi1999}.

\subsection{Proof of feasibility (2000-2010)}

In 2004, DARPA started its series of Grand Challenges to foster the development
of robotics technology. 
The first edition ended without a declared winner, since no contestant manage to
cover even a 10\% of the 240km route including dirt roads and off-road sections.
In 2005 edition, five of the 23 contestants finished the 212 km race. 
The winner was Stanford University racing team. Its vehicle Stanley carried 5 
LiDAR units 
%used to create a 3D map of the environment with special attention to road 
%geometry 
, a frontal camera, GPS sensors, an IMU, wheel odometry and two 
automotive radars \cite{Thrun2006}. These sensors were the input of a 
complex processing pipeline distributed over 6 computers, that involved
perception and estimation techniques, artificial intelligence, 3D mapping, 
risk assessment and path planning.
%, requiring 6 computers to do the processing. 

For the next DARPA Challenge (Urban Challenge, 2007), participants had to
complete a 96 km course in urban area, sharing the road 
with other participants and cars driven by professional drivers. Robots were 
required to obey traffic regulations, avoid obstacles and negotiate 
intersections properly. Carnegie Mellon University team won the contest with 
its vehicle Boss \cite{TartanRacing2005}, featuring a complex perception
system composed by two video cameras, 5 
radars and 13 LiDAR (including a roof mounted unit of the novel Velodyne 64HDL).
A cluster of 10 server blades processed a complex behavioral model
\cite{Urmson2007} for covering all the expected situations.

These events triggered the attention of Google. 
The company hired around 15 scientists from the DARPA challenge, 
including the winners of 2005 and 2007 \cite{Montemerlo2008}, 
\cite{Levinson2011}. Google's (and Waymo's) approach to self-driving vehicles
is largely founded in LiDAR and 3D mapping technologies \cite{Chapell2016}. 
All their vehicles have had a roof-mounted spinning LiDAR: Toyota Prius (2009),
the Firefly prototype (2014) and Chrisler Pacifica (2016-present).

Meanwhile, University of Parma created the spin-off VisLab in 2009. 
As opposed to Google approach, they represent one of the strongest supporters
of artificial vision as the main component of perception systems for AD. 
In 2010 they completed the VisLab Intercontinental Autonomous 
Challenge (VIAC), where four automated vans drove from Italy to China 
\cite{Bertozzi2011}.
The leading vehicle did perception (with cameras and LiDARs), decision and 
control, with some human intervention for selecting the route and managing 
critical situations \cite{Broggi2012}. The 13,000 km long trip included
unmapped areas, degraded dirt roads and different traffic conditions. 
In 2013 the PROUD test put a vehicle with no driver behind the wheel in Parma 
roads for doing urban driving in real traffic \cite{Broggi2013}. 
%Selected sensor configuration was very similar to that used in VIAC three years
%before.
 
\subsection{Race to commercial products (2010-present)}
 
In the last decade the landscape of Automated Driving has been dominated by 
private initiatives that foresee the coming of Level 4 and 5 systems in a few 
years. This vision gave birth to several companies devoted to this end, most of 
which were founded by people coming from the DARPA experience, or hired them to
lead the project \cite{Chapell2016}. 

Examples include nuTonomy, (robotaxi company now acquired by Aptiv) co-founded
by the leader of the MIT team in 2007 DARPA Urban Challenge. Cruise has been
founded by a member of the same MIT team.
Otto was founded by an engineer in Google's Street View project, they completed
a beer delivery service with an automated truck in 2016.
Uber hired up to 50 people from the CMU Robotics Lab for its self-driving car 
project. Zoox robotaxi company is co-founded by a member of the Stanford 
Autonomous Driving team with an expertise in LiDAR automated calibration 
\cite{Levinson2011a}, and Aurora company has a similar story with people from 
Uber, MIT and Waymo \cite{Anderson2013}.

Car manufacturers reacted a bit slower. 
%Stanford and Audi teamed to complete an
%automated ascent to Pikes Peak in 2010 (focused on control and not in
%perception) \cite{Funke2012}, they have not really entered into scene
%until the last five years. 
Some of them started independent research lines, for example
BMW has been testing automation prototypes in roads since 2011 
\cite{Aeberhard2015a}, but in the end most manufacturers have created
coalitions with technological startups, as enumerated later in section
\ref{sec:oem-ad}.

Mercedes-Benz presented the Bertha project in 2013, based on an experimental 
S-Class 500 that drove a 103 km route in automated mode. Exteroceptive 
perception relied in close-to-market sensors (8 radars and 3 video cameras), 
presented new concepts as the \emph{Lanelets} for road representation 
\cite{Ziegler2014} and other innovative solutions \cite{Bender2014} processed
in a heterogeneous computing platform (FPGAs, embedded processors). 

This approach to perception --avoiding LiDARs as expensive and far from mass 
production devices-- has been supported by other companies. As an example,
Mobileye started working in embedded computer vision devices in 1999, and by
2015 their technology was present in more than 25 car brands. Some years ago
they started working in a vision-only approach to Automated Driving 
\cite{Mobileye2018}. 
%Their AI is claimed to hand the car with an ``\emph{assertive driving style}'' 
%that deals with traffic in a much less conservative way than its competitors,
%while being safe \cite{Shalev-shwartz2016, Shalev-Shwartz2017}. 
After testing in real conditions \cite{Edelstein2018}, they presented a demo with an automated Ford equipped just with 12 small monocular cameras for fully automated driving in 2018 \cite{Scheer2018}.

Electric car manufacturer Tesla entered the automated driving scene in 2014.
All their vehicles were equipped with a monocular camera (based on 
Mobileye system) and an automotive radar that gathered data and trained a 
"ghost" self-driving system based on reinforcement learning. In 2017 all 
manufactured Tesla vehicles have the hardware version 2, composed by a frontal 
radar, 12 ultrasonic rangers, and eight cameras that provide 360 
degrees coverage. This sensor set together with nVidia Drive PX2 processing 
technology is claimed to be enough for full Level 5 automated driving
\cite{Hawkins2017}, which will be available for a fee (when ready) through a
software update.

In 2015 VisLab was acquired by Ambarella, a company focused in embedded video 
processing. They are working on low power chips able to process dense disparity
maps from high resolution stereo cameras \cite{Ambarella2018}. 
Its latest demo (Silicon Valley, 2018 \cite{AUVSI2018}) fuse data from 10 stereo
pairs (a total of 20 cameras) for creating a ultra-high resolution 3D scene 
delivering 900 million points per second.
Long range vision relies in a forward facing 4k stereo pair together with a 
single automotive radar for better performance under low light or adverse 
weather conditions. 
%This is a different approach to visual-based perception, 
%since stereo vision aims to get the best of both LiDARs and image processing 
%in 
%a single tool, with additional advantages.

Delphi Automotive completed in 2015 an automated trip between San Francisco and
New York city using a custom Audi Q5 with 10 radars, 6 LiDARs and 3 cameras 
onboard. Two years later they acquired nuTonomy, the first company to deliver a 
robotaxi service in public roads (available in reduced area of Singapore), and
created Aptiv. 
Aptiv presented an automated taxi for CES conference in january 2018, as part
of a 20 vehicle fleet that has been serving a set of routes in Las Vegas for
some months. The taxis have an extensive set of 10 radars and 9 LiDARs embedded 
in the bodywork, plus one camera.

Meanwhile, Waymo had grown a fleet of Chrysler Pacifica minivans that has
self-driven 10 million miles by october 2018. They put efforts on cutting 
prices and improving production scalability. For this purpose, they have created
custom sensors: ``two of the three LiDAR [...] are completely new categories of 
LiDAR'' \cite{Waymoteam2017}. 
The long-range LiDAR is claimed to dynamically zoom into 
objects on the road, letting the vehicle see small objects up to 200 m away. 
This reminds the features of OPA solid state LiDARs (see section
\ref{sec:03-lidar-emerging}): random sampling across the 
scanning area and adaptive resolution. Although no details have been released 
to public, they claim to have reduced production cost of LiDAR sensors to less 
than one tenth in a few years. 

% (which still can be high, since the cost of the
%64-layer Velodyne was over US\$ 75,000 when Waymo started its experiments).
