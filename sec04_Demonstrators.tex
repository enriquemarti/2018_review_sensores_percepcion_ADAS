

This section reviews some of the most relevant technological demonstrations,
contests, challenges and commercial platforms related with automated driving, 
starting from pioneering works in late 1980s until present day. Figure 
\ref{fig:tech-demos} arrange some of the most representative examples in a 
timeline, with a focus on the sensors equipped by each platform.

The timeline serves two different purposes. 
First, it allows to discern different stages ("ages") in the development of
Automated Driving technology. 
Second, it allows to identify trends and approaches to perception for
self-driving vehicles.


%\begin{figure}[h]
%    \centering
%    \includegraphics[width=0.99\textwidth]{"img/AD_Timeline"}
%    \caption{Timeline of relevant technology demonstrators in automated 
%driving}
%    \label{fig:tech-demos}
%\end{figure}

%\begin{table}[H]
%    \caption{Sensing technologies used in relevant AD demos}
%    \label{tab:tech-demos}
%    %\centering
%    \begin{tabularx}{\linewidth}{L r c c c c c c c c}
%%        \toprule
%          &  & \multicolumn{8}{c}{\textbf{Perception technologies}} \\ 
%         \multirow{1}{*}{\textbf{Demo / entity}}   & 
%\multirow{1}{*}{\textbf{Year}} & \rotatebox{90}{\textbf{Magnetic}}	& 
%\rotatebox{90}{\textbf{Inertial}} & \rotatebox{90}{\textbf{GPS}} & 
%\rotatebox{90}{\textbf{Vision}} & \rotatebox{90}{\textbf{Lidar}} & 
%\rotatebox{90}{\textbf{Radar}} & \rotatebox{90}{\textbf{Sonar}} & 
%\rotatebox{90}{\textbf{V2V comms}}  \\
%        \midrule
%        Prometheus (EUREKA, European Initiative)      & 1987  & - & X & - &  X 
%&  - &  - &  - & - \\
%        Cybercar concept INRIA (France)               & 1991  & - & X & X &  X 
%        &  X &  - &  - & - \\
%        VaMP (Bundeswehr Univ. Munich)                & 1994  & - & X & - &  4 
%        &  - &  - &  - & - \\
%        PATH Demo PATH partnership (UC Berkeley)      & 1997  & X & X & - &  X 
%        &  X &  X &  - & X \\
%        DARPA Challenge (Stanley, Stanford AI dept.)  & 2005  & - & X & X &  1 
%&  5 &  2 &  - & - \\
%        DARPA Urban Challenge (Boss, Carnegie Mellon) & 2007  & - & X & X &  2 
%& 13 &  5 &  - & - \\
%        Google "modified Prius"                       & 2009  & - & X & X &  1 
%&  1 &  4 &  - & - \\
%        VIAC Challenge (VisLab, Parma Univ.)          & 2010  & - & X & X &  3 
%&  4 &  - &  - & - \\
%        Audi Pikes Peak                               & 2010  & - & X & X &  - 
%        &  - &  - &  - & - \\
%        Great Cooperative Driving Challenge           & 2011  & - & X & X &  X 
%&  X &  X &  - & X \\
%%        Thunderhill Raceway Park Audi                 & 2012  & ? & ? & ? &  
%%? &  ? &  ? &  ? & - \\
%        PROUD Project   (VisLab, Parma Univ.)         & 2013  & - & X & X &  4 
%&  3 &  - &  - & - \\
%        Bertha Project (Mercedes-Benz)                & 2013  & - & X & X &  3 
%&  - &  8 &  - & - \\
%        Firefly car (Google / Waymo)                  & 2014  & - & X & X &  X 
%&  X &  X &  - & - \\
%        Autopilot 1   (Tesla)                         & 2014  & - & X & X &  1 
%&  - &  1 & 12 & - \\
%        SF-NY coast-2-coast (Delphi)                  & 2015  & - & X & - &  3 
%&  6 & 10 &  - & - \\
%        Silicon Valley - Las Vegas (Audi)             & 2015  & - & X & X &  5 
%        &  2 &  4 &  - & - \\   
%        Budweiser delivery (Uber - Otto)              & 2016  & - & X & X & >4 
%        &  1 &  ? &  ? & - \\
%        Autopilot 2 Hardware (Tesla)                  & 2017  & - & X & X &  8 
%        &  - &  1 & 12 & - \\  
%        Automated Taxi at Las Vegas (Aptiv-Lyft)      & 2018  & - & ? & ? &  1 
%        &  9 & 10 &  - & - \\
%        Enhanced Vehicle Autonomy (Ambarella)         & 2018  & - & ? & ? & 10 
%        &  - &  1 &  - & - \\
%        Mobileye demo at Jerusalem (Mobileye)         & 2018  & - & - & - & 12 
%        &  - &  - &  - & - \\    
%        
%%        \bottomrule
%    \end{tabularx}
%\end{table}


%Pioneer age (1980-2000)
%  - Mostly vision-based technologies
%  - Large project EUREKA and independent research  
%  - PATH demo shows technical feasibility, and opens up to other sensors
%  
%Jump to feasibility (2000-2010)
%  - DARPA 2004-2005 is a fundamental milestone.
%  - Consolidation of major sensor families: video, lidar, radar.
%  - Draw public attention and, most important, private companies attention.
%  - SAE and other "standards" (conceptual frameworks)
%  - Google is first private company showing promising results on AD.
%
%Race to commercial products (2010-present)
%  - Car manufacturers doing R\&D.
%  - Considering production scaling and costs.
%  - Push to create regulation and laws.
%  - Coordinated industry effort to generate an AD ecosystem (HW, SW, 
%  sensors...).
  
  
  
\begin{figure*}[p] %[h]
  \centering
  %\includegraphics[width=\textwidth]{"img/AD_Timeline_2"}
  \includegraphics[width=0.95\textheight,angle=90,keepaspectratio]{"img/AD_Timeline_2"}
  \caption{Timeline of relevant AD demonstrators and its sensor setup for 
      perception}
  \label{fig:tech-demos}
\end{figure*}

\subsection{Pioneer works (1980-2000)}

Pioneer works in true Automated Driving start around mid 1980s. The
Bundeswehr University of Munich developed automated vehicles based on visual 
guidance, as VaMoRs \cite{Dickmanns1987} (a large van) and VaMP 
\cite{Gregor2002} (a Mercedes 500 SEL). Professor Ernst Dickmanns' team 
developed a saccadic vision system based on cameras mounted on a rotating 
platform that allowed cameras to focus in the details considered important.
These system mark a milestone because of the huge amount of computational power 
needed to process the visual information, according to the standards of the 
age. It was accomplished by integrating sixty transputers that executed an
intelligent 4-D approach to object tracking.

In early 1990s Michel Parent creates the concept of a "Cybercar", a small 
automated electric vehicle for shared urban use \cite{Parent1993}. In 1993 
starts project Praxitèle \cite{Massot1999} as a mixed public and industry
initiative that in a few years lead to a fully functional prototype. 
In 1997 the Cybercar debuts in Schippol airport to transport passengers between 
the terminal and long stay parking \cite{Ozguner2007}. 
The vision and LiDAR based vehicle lacked pedals or steering wheels and moved
autonomously in a dedicated lane that included semaphores and some pedestrian 
crossings.

Also in 1997, the National Automated highway System Consortium presented a 
demonstration of Automated Driving functionalities\cite{Thorpe1997}. 
The demo showed road following functionality based on vision sensors, 
distance maintenance based on LiDAR, vehicle following based on Radar
and other funcionalities including cooperative maneuvers and mixed automated 
and manual driving. 
The demo was intended to be a proof of technical feasibility of such 
technologies, creating the foundations for further developments.

Back in that time, relevant functional demonstrators still relied
heavily in visual processing. The University of Parma started in 1996 a
project with its vehicle ARGO \cite{Broggi1998}, a Lancia Thema equipped with 
a two camera vision-based system that allowed road following, platooning and 
obstacle avoidance. They detected lane lines and vehicles using classical image 
processing techniques including preprocessing steps, feature extraction,
model fitting and spatio-temporal filtering.
It completed over 2000 km in autonomous mode in public roads, and the 
full experience was compiled in a book \cite{Broggi1999}.


\subsection{Proof of feasibility (2000-2010)}

However, it was not until 2004 that DARPA started its series of Grand Challenges
to foster the development of robotics technology. 
The first edition consisted in travelleing a 240 km long route comprising dirt 
roads and off-road sections. It ended without a declared winner, since no 
contestant manage to cover even a 10\% of the route.
The next year five of the 23 contestants finished the 212 km race of the 
DAROA Grand Challenge 2005. 
The winner was Stanley, the vehicle from Stanford University racing team, which 
was awarded the 2 million dollar prize.
Stanley carried 5 LiDAR units used to create a 3D map of the environment with 
special attention to road geometry, a frontal camera, GPS sensors, an IMU, 
wheel odometry and two automotive radars \cite{Thrun2006}. This unprecedenting
amount of sensors were the input of a complex processing pipeline that involved 
perception and estimation techniques, artificial intelligence, 3D mapping, risk 
assessment and path planning, requiring 6 computers to do the processing.

The next DARPA Grand Challenge, known as Urban Challenge, took place in 2007.
Participants had to complete a course of 96 km in urban area, while sharing the
road with other participants and cars driven by professional drivers. Robots
were required to obey traffic regulations, avoid obstacles and negotiate 
intersections properly. Carnegie Mellon University team, under the lead of
Chris Urmson, won the contest with 
its vehicle Boss \cite{TartanRacing2005}, a 
modified Chevy Tahoe that included two video cameras, 
5 radars and 13 LiDAR --including a roof mounted unit of the novel Velodyne
64HDL. Data was processed in a cluster of 10 server blades and featured
a complex behavioral model \cite{Urmson2007} for covering all the expected
situations.

These events had a huge scientific impact, but also triggered the attention of
Google, which in 2009 put Sebastian Thrun (former director of Stanford AI 
department, winners of 2005 Challenge and 2nd classified in 2007) at the 
head of its secret self-driving project. 
The company hired around 15 more scientist including Michael Montemerlo 
\cite{Montemerlo2003a} (leader of Stanford team in 2007 \cite{Montemerlo2008}) 
and Chris Urmson.
Google vehicles have always relied in a roof-mounted spinning lidar as 
a cornerstone of their perception systems. All their vehicle have used one, 
starting with the original 2009 Toyota Prius, continuing with 2014 Firefly
prototype and ending with Waymo's Chrisler Pacifica (2016-present).
Thrun, an expert in robotics and probabilistic localization algorithms
\cite{Levinson2011}, 
was also behind the development of Google Street View since 2007. 
This has probably influenced in Google's approach to self-driving vehicles 
largely founded in 3D mapping technologies \cite{Chapell2016} for which LiDARs
are a fundamental tool.
 
 
Meanwhile, Broggi's team from University of Parma created the spin-off VisLab
in 2009. They continued creating new pieces of technology and preparing 
outstanding Automated Driving demonstrators. As opposed to Google approach,
they represent one of the strongest supporters of artificial vision as the main 
component of perception systems for AD. In 2010 they 
completed the VisLab Intercontinental Autonomous Challenge (VIAC), where
four automated vans drove from Italy to China \cite{Bertozzi2011}.
The leading vehicle did perception, decision and control with some human 
intervention for selecting the route and managing critical situations
\cite{Broggi2012a}, and the rest of the vehicles followed it based on visual
tracking and GPS points \cite{Broggi2012a}. 
The 13,000 km long trip included unmapped areas, degreaded dirt roads and
different traffic conditions. Leading vehicle employed several LiDARs and
video cameras, with a strong focus on stereo vision \cite{Broggi2011}.
Three years later, the PROUD test put a vehicle with no driver behind the
wheel in Parma roads for doing urban driving in real traffic 
\cite{Broggi2013}. 
It used a perception scheme somehow similar to VIAC configuration, but more 
based in video and less in LiDARs.
 
\subsection{Race to commercial products (2010-present)}
 
In the last decade, the landscape of Automated Driving has been dominated
by private initiatives. Automated vehicles between Levels 4 and 5 appear
as a possibility in a few years, giving birth to several companies devoted to 
this end. 
A significant number of them have been founded by people coming from the 
DARPA experience, or have hired them to lead the project 
\cite{Chapell2016}.
 For
example, robotaxi company nuTonomy (now acquired by Aptiv) was co-founded
by Emilio Frazzoli who lead the MIT team that ended \#4 in 2007 DARPA 
Urban Challenge.
Cruise has been founded by Kyle Vogt, an undergrad student in
Frazzoli's team back in 2007. Otto, a company that pursues self-driving 
truck --they completed a beer delivery service with an automated truck 
in 2016--, is founded by Anthony Lewandosky, one of the co-creators of
the only two-wheeled vehicle that has participated in DARPA challenges
and later an engineer in Google's Street View project.
Uber nuttered its self-driving car project with up to 50 people from the CMU
Robotics Lab.
Zoox robotaxi company is co-founded by Jesse Levinson, former Stanford doctorate
at the Autonomous Driving team with an expertise in LiDAR automated calibration
\cite{Levinson2011a}.
Aurora company is founded by Sterling Anderson, who developed an intelligent 
copilot \cite{Anderson2013} while doing its PhD in MIT and was later hired by
Tesla as head engineer of its Autopilot system, Drew Bagnell who directed 
autonomy and perception team at Uber, and Chris Urmson.

Car manufacturers have reacted a bit slower. Apart from collaborations with
research entities, as Stanford-Audi Pikes Peak ascent in 2010 that was
focused on control close to dynamic limits and did not any actual perception
\cite{Funke2012}, 
they have not really entered into scene until the last five years or less. 
Most of them have ended creating coalitions with technological startups, as 
enumerated later in section \ref{sec:oem-ad}.

Mercedes-Benz presented the Bertha project in 2013. Bertha is an experimental
S-Class 500 that drove a 103 km route in automated mode. They followed a
rational manufacturer approach and used only close-to-market sensors.
The vehicle was eqquiped with 8 radars and 3 video cameras for exteroceptive 
perception.
This work is an exception in the sense that it is an actual piece of research
with published results. The team presented innovative solutions to perception 
\cite{Bender2014} with efficient algorithms processed in an heterogeneous 
computing platform (FPGAs, embedded processors) and new concepts as the 
Lanelets for road representation \cite{Ziegler2014}.

This approach to perception --avoiding LiDARs as expensive and far from mass
production devices-- has been supported by other companies. An interesting 
example is Mobileye, the Israel-based company co-founded in 1999 by Amnon 
Shashua (MIT doctorate and AI professor in Hebrew University of Jerusalem).
They started working in embedded computer vision devices and by 2015 their 
technology was present in more than 25 car brands. Some years before they
started working in a completely vision-based approach to perception in 
self-driving vehicles \cite{Mobileye2018}. Strongly based in formal 
demonstrations 
\cite{Shalev-Shwartz2017}, their AI is claimed to hand the car with an 
"assertive driving" style that deals with traffic in a much less
conservative way than its competitors, while being safe 
\cite{Shalev-shwartz2016}. 
Ater some time testing their cars in Jerusalem and Tel Aviv
\cite{Edelstein2018}, in 2018 they presented a demo with an automated Ford 
equipped just with 12 small monocular cameras for fully automated driving 
\cite{Scheer2018}.

Tesla is an electric car manufacturer that entered the self-driving scene back
in 2014. All their vehicles were eqquiped with a monocular camera (actually
Mobileye system) and an automotive radar that gathered data and trained a
"ghost" self-driving system based on reinforcement learning. In 2017 all 
manufactured Tesla vehicles have the version 2 hardware, composed by an
automotive frontal radar, the same 12 ultrasonic rangers, and eight cameras 
that provide 360 degrees coverage. This sensor set together with nVidia
Drive PX2 processing technology is claimed to be enough for full Level 5 
automated driving, which will be available for a fee when ready --just a 
software update.

In 2015 VisLab was acquired by Ambarella, a company focused in embedded video
processing. With their industrial support, they have achieved to manufacture
low power chips able to process dense disparity maps from high resolution 
stereo cameras \cite{Ambarella2018}.
Its latest demo took place in Silicon Valley in 2018 \cite{AUVSI2018},
and does perception with 10 stereo pairs (a total of 20 cameras) for creating a
ultra high resolution 3D scene, delivering around 900 million ponts per second.
Long range vision relies in a forward facing 4k stereo pair together with a 
single automotive radar for better performance under low light or adverse
weather conditions. This is a different approach to visual-based perception,
since stereo vision aims to get the best of both LiDARs and image processing in
a single tool, with additional advantages.

The debate around price and production scalability for some sensors is still
an open issue. Back to Waymo project, the Chrysler Pacifica equipps custom 
sensors, "two of the three LiDAR [...] are actually 
completely new categories of LiDAR" \cite{Waymoteam2017}. The long range 
LiDAR is claimed to dynamically zoom into objects on the road, letting the
vehicle see small objects up to 200 m away. 
This  reminds the features of solid state LiDARs: random sampling the scanning 
area and adaptive resolution. Waymo is continuously growing its fleet, ordering 
100 units in may 2016 and 62,000 more vehicles in june 2018, demonstrating some 
degree of production scalability. 
Although no details have been released to public, they claim to have reduced
production cost of LiDAR sensors to less than one tenth in a few years 
(which still can be high, since original 64-layer Velodyne costed over 
US\$ 75,000 when Waymo started its experiments).


\emph{FALTA: Delphi Coast 2 Coast}

