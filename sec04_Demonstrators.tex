
En esta sección revisamos demos relevantes.
IDEA: detallar para cada demostrador el camino completo "qué hace -> competencias/habilidades -> información usada -> sensores"

This section reviews some of the most relevant technological demonstrations,
contests, challenges and commercial platforms related with automated driving, 
starting from pioneering works in late 1980s until present day. The review is
based on a timeline focused on the sensors equipped by each platform.

Table \ref{tab:tech-demos} condenses the raw data displayed in Figure 
\ref{fig:tech-demos} timeline. The timeline serves two different purposes. 
First, it allows to discern different stages ("ages") in Automated Driving 
technology. Second, it allows to identify trends and approaches to perception 
for self-driving vehicles.


%\begin{figure}[h]
%    \centering
%    \includegraphics[width=0.99\textwidth]{"img/AD_Timeline"}
%    \caption{Timeline of relevant technology demonstrators in automated 
%driving}
%    \label{fig:tech-demos}
%\end{figure}

\begin{figure}[h]
\centering
%\includegraphics[width=\textwidth]{"img/AD_Timeline_2"}
\includegraphics[width=0.95\textheight,angle=90,keepaspectratio]{"img/AD_Timeline_2"}
\caption{Timeline of sensor setup for relevant technology demonstrators in AD}
\label{fig:tech-demos}
\end{figure}

\begin{table}[H]
    \caption{Sensing technologies used in relevant AD demos}
    \label{tab:tech-demos}
    %\centering
    \begin{tabularx}{\linewidth}{L r c c c c c c c c}
%        \toprule
          &  & \multicolumn{8}{c}{\textbf{Perception technologies}} \\ 
         \multirow{1}{*}{\textbf{Demo / entity}}   & \multirow{1}{*}{\textbf{Year}} & \rotatebox{90}{\textbf{Magnetic}}	& \rotatebox{90}{\textbf{Inertial}} & \rotatebox{90}{\textbf{GPS}} & \rotatebox{90}{\textbf{Vision}} & \rotatebox{90}{\textbf{Lidar}} & \rotatebox{90}{\textbf{Radar}} & \rotatebox{90}{\textbf{Sonar}} & \rotatebox{90}{\textbf{V2V comms}}  \\
        \midrule
        Prometheus (EUREKA, European Initiative)      & 1987  & - & X & - &  X &  - &  - &  - & - \\
        Cybercar concept INRIA (France)               & 1991  & - & X & X &  X 
        &  X &  - &  - & - \\
        VaMP (Bundeswehr Univ. Munich)                & 1994  & - & X & - &  4 
        &  - &  - &  - & - \\
        PATH Demo PATH partnership (UC Berkeley)      & 1997  & X & X & - &  X 
        &  X &  X &  - & X \\
        DARPA Challenge (Stanley, Stanford AI dept.)  & 2005  & - & X & X &  1 &  5 &  2 &  - & - \\
        DARPA Urban Challenge (Boss, Carnegie Mellon) & 2007  & - & X & X &  2 & 13 &  5 &  - & - \\
        Google "modified Prius"                       & 2009  & - & X & X &  1 &  1 &  4 &  - & - \\
        VIAC Challenge (VisLab, Parma Univ.)          & 2010  & - & X & X &  3 &  4 &  - &  - & - \\
        Audi Pikes Peak                               & 2011  & - & X & X &  - &  - &  - &  - & - \\
        Great Cooperative Driving Challenge           & 2011  & - & X & X &  X &  X &  X &  - & X \\
%        Thunderhill Raceway Park Audi                 & 2012  & ? & ? & ? &  ? &  ? &  ? &  ? & - \\
        PROUD Project   (VisLab, Parma Univ.)         & 2013  & - & X & X &  4 &  3 &  - &  - & - \\
        Bertha Project (Mercedes-Benz)                & 2013  & - & X & X &  3 &  - &  8 &  - & - \\
        Firefly car (Google / Waymo)                  & 2014  & - & X & X &  X &  X &  X &  - & - \\
        Autopilot 1   (Tesla)                         & 2014  & - & X & X &  1 &  - &  1 & 12 & - \\
        SF-NY coast-2-coast (Delphi)                  & 2015  & - & X & - &  3 &  6 & 10 &  - & - \\
        Silicon Valley - Las Vegas (Audi)             & 2015  & - & X & X &  5 &  2 &  4 &  - & - \\   
        Autopilot 2 (Tesla)                           & 2016  & - & X & X &  8 &  - &  1 & 12 & - \\  
        Budweiser delivery (Uber - Otto)              & 2016  & - & X & X & >4 &  1 &  ? &  ? & - \\
        Automated Taxi at Singapore (NuTonomy)        & 2017  & - & ? & ? &  ? &  ? &  ? &  ? & - \\    
%        \bottomrule
    \end{tabularx}
\end{table}

Pioneer work by Ernst Dickmanns with VaMoRs a Mercedes van featuring 
visual-based guidance \cite{Dickmanns1987}.

Eureka Prometheus project extended for 8 years (1987-1995), and has been the 
largest R\&D project. One of the vehicles involved was VaMP \cite{Gregor2002}, 
a Mercedes 500 SEL reengineered by the Bundeswehr University of Munich that
relied only in vision for guiding the vehicle. In 1994 it completed a trip over 
1000 km long in highways under normal traffic conditions, performing lane 
changes and adapting its speed according to road characteristics and surrouding 
vehicles.
This demonstration was one of the final events of the PROMETHEUS project.
The system relied in two movables platforms with two cameras each, that 
implemented a saccadic vision system. The platforms rotated 
to allow cameras focus in the details considered important.
This system marks a milestone because of the huge amount of computational power 
needed to process the visual information. It was accomplished by integrating
sixty transputers that did the work.

Cybercar INRIA: pedir referencias a Joshue.

University of Parma's ARGO \cite{Broggi1998}. A Lancia Thema equipped with a
vision-based system that allowed road following and obstacle avoidance.

Later on in 1997, the National Automated highway System Consortium presented a 
demonstration of Automated Driving functionalities\cite{Thorpe1997}. 
The demo showed road following functionality based on vision sensors, 
distance maintenance based on LiDAR, vehicle following based on Radar
and other funcionalities including cooperative maneuvers and mixed automated 
and manual driving. 
The demo was intended to be a proof of technical feasibility of such 
technologies, creating the foundations for further developments.

However, it was not until 2004 that DARPA started its series of Grand Challenges
to foster the development of robotics technology. 
The first edition consisted in travelleing a 240 km long route comprising dirt 
roads and off-road sections. It ended without a declared winner, since no 
contestant manage to cover even a 10\% of the route.
The next year five of the 23 contestants finished the 212 km race. 
The winner was Stanley, the vehicle from Stanford University racing team, which 
was awarded the 2 million dollar prize.
Stanley carried 5 LiDAR units used to create a 3D map of the environment with 
special attention to road geometry, a frontal camera, GPS sensors, an IMU, 
wheel odometry and two automotive radars \cite{Thrun2006}. This unprecedenting
amount of sensors were the input of a complex processing pipeline that involved 
perception and estimation techniques, artificial intelligence, 3D mapping, risk 
assessment and path planning, requiring 6 computers to do the processing.

The next DARPA Grand Challenge, known as Urban Challenge, took place in 2007.
Participants had to complete a course of 96 km in urban area, while sharing the
road with other participants and cars driven by professional drivers. Robots
were required to obey traffic regulations, avoid obstacles and negotiate 
intersections properly. Carnegie Mellon University won the contest with 
its vehicle Boss (http://www.tartanracing.org/press/boss-glance.pdf), a 
modified Chevy Tahoe that included two video cameras, 
5 radars and 13 LiDAR --including a roof mounted unit of the novel Velodyne
64HDL. Data was processed in a cluster of 10 server blades and featured
a complex behavioral model \cite{Urmson2007} for covering all the expected
situations, which included executing U-turns or ....



%Useful references:
%\begin{itemize}
%    \item Prometheus: 
%    \item INRIA Cybercar:
%    \item PATH Demo: 
%\url{http://www.path.berkeley.edu/sites/default/files/documents/intel63.pdf} 
%(DUDA: se habla de "laser radars"... sera lidar)
%    \item DARPA05 Stanford "Stanley":  
%\url{https://www-cs.stanford.edu/people/dstavens/jfr06/thrun_etal_jfr06.pdf}  
%    \item DARPA07 CMU "Boss":  
%\url{http://www.tartanracing.org/press/boss-glance.pdf} 
%\url{http://www.andrew.cmu.edu/user/mnclark/(14)Clark.pdf}
%    \item Google autonomous vehicles: 
%    \item VIAC vehicles:  \url{http://viac.vislab.it/?page_id=159}
%    \item Bertha project:
%    \item Tesla autopilot: wiki
%    \item Delphi coast2coast: 
%    \url{https://newatlas.com/delphi-drive-completed/36859/}
%    \item 
%\end{itemize}
%
%Contests:
%\begin{itemize}
%    \item DARPA challenges (2004, 2005, 2007)
%    \item SAE AutoDrive challenge (2019)
%    \item Great Cooperative Driving Challenge (2011, 2016)
%    \item SEAT Autonomous Driving Challenge (2018)
%    \item Audi Autonomous Driving Cup (2015, 2016, 2017, 2018)
%    \item Udacity Challenges:
%        \begin{itemize}
%            \item Didi's Open-Source Autonomous Vehicle (2017)
%            \item Bosch's Path Planning Challenge (2018)
%            \item Lyft's Perception Challenge (2018)
%        \end{itemize}
%    \item CVPR Video Segmentation Challenge (2018)
%\end{itemize}
