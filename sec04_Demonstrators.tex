This section describes some of the most relevant technological demonstrations, competitions, challenges and commercial platforms related with automated driving, starting from pioneering works in late 1980s until present day. Figure \ref{fig:tech-demos} shows the timeline, with the focus on the sensors equipped by each platform.

The timeline allows to discern different stages ("ages") in the development of Automated Driving technology, and to identify trends and approaches from the perception point of view for Automated Vehicles.

\begin{figure*}[p] %[h]
  \centering
  %\includegraphics[width=\textwidth]{"img/AD_Timeline_2"}
  \includegraphics[width=0.95\textheight,angle=90,keepaspectratio]{"img/AD_Timeline_2"}
  \caption{Timeline of relevant AD demonstrators and its sensor setup for 
      Perception systems}
  \label{fig:tech-demos}
\end{figure*}

\subsection{Pioneer works (1980-2000)}

Pioneer works in Automated Driving start around mid-1980s. The Bundeswehr 
University of Munich developed automated vehicles based on visual guidance, as 
VaMoRs \cite{Dickmanns1987} (a large van) and VaMP \cite{Gregor2002} (a 
Mercedes 500 SEL). Professor Ernst Dickmanns' team developed a saccadic vision 
system based on cameras mounted on a rotating platform that allowed cameras to 
focus in the details considered important.
These systems mark a milestone because of the huge amount of computational 
power needed to process the visual information, according to the standards at 
that moment. It was accomplished by integrating sixty transputers that executed 
an intelligent 4-D approach to object tracking.

In early 1990s INRIA creates the concept of a "Cybercar", a small automated 
electric vehicle for shared urban use \cite{Parent1993}. In 1993 starts project 
Praxitèle \cite{Massot1999} as a mixed public and industry initiative that in a 
few years lead to a fully functional prototype. 
In 1997 the Cybercar debuts in Schippol airport to transport passengers, between the terminal and long stay parking \cite{Ozguner2007}. The vision and LiDAR based vehicle lacked pedals or steering wheels and moved autonomously in a dedicated lane that included semaphores and some pedestrian crossings.

Also in 1997, the National Automated Highway System Consortium presented a 
demonstration of Automated Driving functionalities\cite{Thorpe1997}. The demo 
showed road following functionality based on vision sensors, distance 
maintenance based on LiDAR, vehicle following based on Radar and other 
functionalities including cooperative maneuvers and mixed environments. The 
demo was intended to be a proof of technical feasibility of such technologies, 
creating the foundations for further developments. 

At that time, relevant functional demonstrators strongly relied on visual 
processing. The University of Parma started in 1996 a project with its vehicle 
ARGO \cite{Broggi1998}, a Lancia Thema equipped with two cameras that allowed 
road following, platooning and obstacle avoidance. They detected lane lines and 
vehicles using classical image processing techniques including preprocessing 
steps, feature extraction, model fitting and spatio-temporal filtering.
It completed over 2000 km of autonomous driving in public roads, and the full 
experience was compiled in a book \cite{Broggi1999}.

\subsection{Proof of feasibility (2000-2010)}

However, it was not until 2004 that DARPA started its series of Grand 
Challenges to foster the development of robotics technology. The first edition 
consisted in travelling a 240 km long route comprising dirt roads and off-road 
sections. It ended without a declared winner, since no contestant manage to 
cover even a 10\% of the route. The next year five of the 23 contestants 
finished the 212 km race of the DARPA Grand Challenge 2005. The winner was 
Stanley, the vehicle from Stanford University racing team. Stanley carried 5 
LiDAR units used to create a 3D map of the environment with special attention 
to road geometry, a frontal camera, GPS sensors, an IMU, wheel odometry and two 
automotive radars \cite{Thrun2006}. These sensors were the input of a 
complex processing pipeline that involved perception and estimation techniques, 
artificial intelligence, 3D mapping, risk assessment and path planning, 
requiring 6 computers to do the processing. 

The next DARPA Grand Challenge, known as Urban Challenge, took place in 2007. Participants had to complete a course of 96 km in urban area, while sharing the road with other participants and cars driven by professional drivers. Robots were required to obey traffic regulations, avoid obstacles and negotiate intersections properly. Carnegie Mellon University team won the contest with its vehicle Boss \cite{TartanRacing2005}, a modified Chevy Tahoe that included two video cameras, 5 radars and 13 LiDAR --including a roof mounted unit of the novel Velodyne 64HDL. Data was processed in a cluster of 10 server blades and featured a complex behavioral model \cite{Urmson2007} for covering all the expected situations.

These events had a huge scientific impact, but also triggered the attention of 
Google. The company hired around 15 scientists from the DARPA challenge, 
including the winners of 2005 and 2007 \cite{Montemerlo2008}, 
\cite{Levinson2011}. Google vehicles have always relied in a roof-mounted 
spinning lidar as a cornerstone of their perception systems, starting with the 
original Toyota Prius (2009), the Firefly prototype (2014) and ending with 
Waymo's Chrisler Pacifica (2016-present).
Google's approach to self-driving vehicles is largely founded in 3D mapping technologies \cite{Chapell2016}, where LiDARs are the fundamental tool.

Meanwhile, University of Parma created the spin-off VisLab in 2009. They 
continued creating new pieces of technology and preparing outstanding Automated 
Driving demonstrators. As opposed to Google approach, they represent one of the 
strongest supporters of artificial vision as the main component of perception 
systems for AD. In 2010 they completed the VisLab Intercontinental Autonomous 
Challenge (VIAC), where four automated vans drove from Italy to China 
\cite{Bertozzi2011}.
The leading vehicle did perception (with cameras and LiDARs), decision and 
control, with some human intervention for selecting the route and managing 
critical situations \cite{Broggi2012a}, and the rest of the vehicles followed 
it based on visual tracking and GPS points \cite{Broggi2012a}. The 13,000 km 
long trip included unmapped areas, degraded dirt roads and different traffic 
conditions. 

Three years later, the PROUD test put a vehicle with no driver behind the wheel in Parma roads for doing urban driving in real traffic \cite{Broggi2013}. It used a perception scheme similar to VIAC configuration, but more based in video and less in LiDARs.
 
\subsection{Race to commercial products (2010-present)}
 
In the last decade, the landscape of Automated Driving has been dominated by private initiatives. Automated vehicles between Levels 4 and 5 appear as a possibility in a few years, giving birth to several companies devoted to this end. 

A significant number of them have been founded by people coming from the DARPA experience, or have hired them to lead the project \cite{Chapell2016}. An example is the robotaxi company nuTonomy (now acquired by Aptiv) was co-founded by the leader of the MIT team that ended \#4 in 2007 DARPA Urban Challenge.
Cruise has been founded by a member of the same MIT team in 2007. Otto, a 
company that pursues self-driving truck --they completed a beer delivery 
service with an automated truck in 2016--, is founded an engineer in Google's 
Street View project.
Uber nuttered its self-driving car project with up to 50 people from the CMU 
Robotics Lab. Zoox robotaxi company is co-founded by a member of the Stanford 
Autonomous Driving team with an expertise in LiDAR automated calibration 
\cite{Levinson2011a}. Aurora company has an similar history 
\cite{Anderson2013}, with people from Uber, MIT and Waymo. 

Car manufacturers have reacted a bit slower. Apart from collaborations with research entities, as Stanford-Audi Pikes Peak ascent in 2010 that was focused on control close to dynamic limits and did not any current perception \cite{Funke2012}, they have not really entered into scene until the last five years. Most of them have ended creating coalitions with technological startups, as enumerated later in section \ref{sec:oem-ad}.

Mercedes-Benz presented the Bertha project in 2013, based on an experimental 
S-Class 500 that drove a 103 km route in automated mode. The vehicle was 
equipped exclusively with close-to-market sensors, and used 8 radars and 3 
video cameras for exteroceptive perception.
This work presented innovative solutions in the perception stage 
\cite{Bender2014}, with efficient algorithms processed in a heterogeneous 
computing platform (FPGAs, embedded processors) and new concepts as the 
\emph{Lanelets} for road representation \cite{Ziegler2014}. 

This approach to perception --avoiding LiDARs as expensive and far from mass 
production devices-- has been supported by other companies. An interesting 
example is Mobileye, the Israel-based company started working in embedded 
computer vision devices in 1999, and by 2015 their technology was present in 
more than 25 car brands. Some years before they started working in a completely 
vision-based approach to perception in Automated Vehicles \cite{Mobileye2018}. 
Their AI is claimed to hand the car with an "assertive driving" style that 
deals with traffic in a much less conservative way than its competitors, while 
being safe \cite{Shalev-shwartz2016}, \cite{Shalev-Shwartz2017}. 
After testing in real conditions \cite{Edelstein2018}, they presented a demo with an automated Ford equipped just with 12 small monocular cameras for fully automated driving in 2018 \cite{Scheer2018}.

In 2014, the electric car manufacturer Testa entered in the automated driving scene. All their vehicles were equipped with a monocular camera (based on the Mobileye system) and an automotive radar that gathered data and trained a "ghost" self-driving system based on reinforcement learning. In 2017 all manufactured Tesla vehicles have the version 2 hardware, composed by a frontal radar, the same 12 ultrasonic rangers, and eight cameras that provide 360 degrees coverage. This sensor set together with nVidia Drive PX2 processing technology is claimed to be enough for full Level 5 automated driving, which will be available for a fee when ready --just a software update.

In 2015 VisLab was acquired by Ambarella, a company focused in embedded video 
processing. With their industrial support, they have achieved to manufacture 
low power chips able to process dense disparity maps from high resolution 
stereo cameras \cite{Ambarella2018}. Its latest demo took place in Silicon 
Valley in 2018 \cite{AUVSI2018}, and does perception with 10 stereo pairs (a 
total of 20 cameras) for creating a ultra-high resolution 3D scene. The system
delivers around 900 million points per second.
Long range vision relies in a forward facing 4k stereo pair together with a single automotive radar for better performance under low light or adverse weather conditions. This is a different approach to visual-based perception, since stereo vision aims to get the best of both LiDARs and image processing in a single tool, with additional advantages.

The debate around price and production scalability for some sensors is still an 
open issue. Back to Waymo project, the Chrysler Pacifica equips custom sensors, 
"two of the three LiDAR [...] are completely new categories of LiDAR" 
\cite{Waymoteam2017}. The long-range LiDAR is claimed to dynamically zoom into 
objects on the road, letting the vehicle see small objects up to 200 m away. 
This reminds the features of solid state LiDARs: random sampling across the 
scanning area and adaptive resolution. Although no details have been released 
to public, they claim to have reduced production cost of LiDAR sensors to less 
than one tenth in a few years (which still can be high, since original 64-layer 
Velodyne costed over US\$ 75,000 when Waymo started its experiments).

Another important developer is Delphi, which completed in 2015 an automated
trip between San Francisco and New York city using a custom Audi Q5 with
10 radars, 6 LiDARs and 3 cameras onboard. Sensors were integrated in the
bodywork, giving the impression of being a regular vehicle. In 2017 Delphi
acquired nuTonomy, the first company to deliver a robotaxi service 
in public roads (available in reduced area of Singapore), and created Aptiv. 
Aptiv presented an automated taxi for CES 
conference in january 2018, as part of a 20 vehicle fleet that has been 
serving a set of routes in Las Vegas for some months. The taxis have an
extensive set of 10 radars and 9 LiDARs also embedded in the bodywork, plus
one camera.
