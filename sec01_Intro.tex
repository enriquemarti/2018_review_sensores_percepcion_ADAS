%Hacer referencia a los accidentes, o plantear otra motivación?
Each year, milion of people die on road accidents, and millions more get injured. In addition to the social cost, it also causes an important economic cost for many nations. A high percentage of those accidents are caused due to a driver mistake. Falling asleep or being fatigated, being under the alcohol or drugs effects or just an inmature driving are some of the causes of all those accidents. 
In order to prevent all those driving mistakes, vehicles are equiped with safety systems that are more sofisticated with time. Many vehicles companies are researching into vehicles more and more automated towards completly self-driving vehicles, in order to eliminate those mistakes made by the driver.
%Falta completar con datos reales y referencias.

Based on recent developments and demonstrations around the world, there is a tendency to think that Automated Driving with a high level of automation will be available in a few years. 
Highly advanced ADAS, like Adaptative Cruise Control (ACC), Automatic Emergency Braking(AEB) or Lane Keep Asistant (LKA) are currently avalible at the market and highly accepted between users. Also, some recent demos like Waymo, Tesla, and more (discused with more detail in Section IV) shows how the autonomous vehicles in the future will be.
However, there are still many research challenges, such as navigation in urban dynamic environments, accurate obstacle avoidance capabilities, environment understanding in real-time, and perception uncertainties among others. This research is needed to allow cooperative maneuvers between automated and semiautomated vehicles, which still need further efforts in real implementation, specifically in urban environment. 

The architecture of autonomous vehicles is usually divided into three categories: Perception of the environment, behaviour planning and motion execution[Sagar Behere](sense-plan-act). The first category, perception of the environment, can be challenging for complex and dynamic environments. Autonomous vehicles obtain data of their sorrounding making use of different sensors, such as cameras, LiDARS and radars. This raw data is then processed to extract relevant characteristics which are the input of the following categories (behaviour planning and motion execution), that will perform tasks such as path planning, collision avoidance or control of the vehicle in between many[Bagloee2016].

This paper pays special attention to the sensor and perception stages. Automated vehicles need to have a good understanding of the surrounding environment in order to drive safely in the roads. A state of the art of the different configurations and sensors used by different manufactrers, based on the most relevant demonstrators on Automated Driving, developed by research institutions and manufactures is presented. This topic has been widely investigated in the literature, but based usually on ADAS implementation. It is a key aspect in the future developments highly automated cars, where Real Time Motion planning needs more accurate and robust inputs. This paper makes an overview and analysis of the main problems, applications and sensor technologies available in the market.

The content of this paper is organized as follows. Section II describes the sensors commonly used for perception explaining the technologies that they use and the new emerging ones that will be used in the future. The most important competences in perception are discused in section III, describing the different sensors that can used in each competence and discussing their advantages and disadvantages. Section IV presents the most relevant works and demos, describing the kind of sensors used for each one, followed by a review of the commercial sensors systems avalible in the market in section V. Finally, in section VI, a final discussion of the future challenges and implantation challenges related with the autonomous vehicle's sensors is presented. 
