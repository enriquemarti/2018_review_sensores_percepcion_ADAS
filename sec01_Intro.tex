
Based on recent developments and demonstrations around the world, there is a tendency to think that Automated Driving with a high level of automation will be available in few years. However, there are still some research challenges, such as navigation in urban dynamic environments, accurate obstacle avoidance capabilities, environment understanding in real-time, and perception uncertainties among others. This research is needed to allow cooperative maneuvers between automated and semiautomated vehicles, which still need further efforts in real implementation, specifically in urban environment. 

This paper pays special attention to the sensor and perception stages for automated vehicles. A state of the art, based on the most relevant demonstrators on Automated Driving developed by research institutions and manufactures is presented. This topic has been widely investigated in the literature, but based main on ADAS implementation. It is a key aspect in the future developments highly automated cars, where Real Time Motion planning needs more accurate and robust inputs. In this paper makes an overview of the main problems, application and sensor technologies available in the market.

The rest of the paper is organized as follows. In Section ...
