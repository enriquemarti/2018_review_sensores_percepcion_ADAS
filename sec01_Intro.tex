%Hacer referencia a los accidentes, o plantear otra motivación?
Every year more than one milion people die on road accidents and several 
million more get injured. In addition to the social cost, it also has an
important economic impact for nations worldwide. According to 
\cite{Thomas2013} the most frequent causes for car accidents in the
European Union are human related: speeding, driving under the effects of
alcohol or drugs, reckless driving, distractions or just plain misjudgments.
Automated Driving systems, also called self-driving vehicles, aim to take the 
human driver out of the equation. Thus, they are designed to be a valuable tool
towards reducing the number of traffic accidents.
%Falta completar con datos reales y referencias.

Based on recent developments and demonstrations around the world, there is a 
tendency to think that Automated Driving with a high level of automation will 
be available in a few years. 
Advanced Driving Asistance Systems (ADAS), like Adaptative Cruise Control 
(ACC), Automatic Emergency Braking (AEB) or Lane Keep Asistant (LKA) are
currently avalible at the market and highly accepted among users. 
Also, some recent demos like Waymo or Aptiv 
(discused in detail in Section \ref{sec:04-relevantdemos}) shows how future 
autonomous vehicles will be.
However, there are still many research challenges, such as navigation in urban 
dynamic environments, accurate obstacle avoidance capabilities, environment 
understanding in real-time, and perception uncertainties among others.

\emph{Further research is needed to allow cooperative maneuvers between 
automated and semiautomated vehicles, which still need further efforts in
    real implementation, specifically in urban environment.}

The architecture of autonomous vehicles is usually divided into three 
categories: Perception of the environment, behaviour planning and motion 
execution \ref{behere2015functional}. Autonomous 
vehicles obtain data of their sorrounding making use 
of different sensors, such as cameras, LiDARS and radars. This raw data is then 
processed to extract relevant characteristics which are the input of the 
following stages (behaviour planning and motion execution), that will 
perform tasks such as path planning, collision avoidance or control of the 
vehicle among others. 

Perception is a very challenging problem for several
reasons. First, the environment is complex and highly dynamic, with some cases
involving a large number of participants (dense traffic, populated cities). 
Second, it needs to work reliably under a wide range of external conditions, 
including lighting and weather (rain, fog, snow, dust). 
Perception errors are propagated and can be the cause of severe accidents. 
Some real examples include the 2016 Tesla AutoPilot accident \cite{NTSB2017},
where a man was killed after its car crashed a truck: 
the camera failed to detect the truck because
it was painted in a color similar to the bright sky, while at the same time 
the radar detection was descarded as background noise by perception algorithms.
Later in 2018, other Tesla vehicles have crashed against highway dividers
after the lane following system failed to detect faded lines.
Also in 2018, an experimental self-driving Uber vehicle killed a woman
crossing the road \cite{NTSB2018} in the night, dressed in dark clothes. 
Only the LiDAR provided a positive detection, that was discarded as a false
positive by perception algorithms.

This article reviews sensor technologies, perception algorithms and explores
their relation, to provide an integral view of the process that leads from raw
sensor data to meaningful information for the driving task.
This topic has been previously investigated in the literature, but usually
centered on ADAS implementation \cite{Yenkanchi2016,Ziebinski2016a} or at a
more general level within Automated Driving \cite{Pendleton2017}. 

%The presented work It presents a review of the most relevant
%sensor technologies for exteroceptive perception, and a state of the art of 
%perception algorithms for generating by the , based on 
%the most relevant demonstrators on Automated Driving, developed by research 
%institutions and manufactures is presented.  It is 
%a key aspect in the future developments highly automated cars, where Real Time 
%Motion planning needs more accurate and robust inputs. This paper makes an 
%overview and analysis of the main problems, applications and sensor 
%technologies available in the market.

The content of the article is organized as follows. Section 
\ref{sec:02-sensors} describes the sensors commonly used for perception 
explaining the technologies, drawbacks and advantages, and related emerging 
technologies that can be used in the future. The section also defines a 
taxonomy of information that will allow to link sensor technologies with the 
next section.
Section \ref{sec:03-problemsapplications} starts describing the most important 
competences in perception, to proceed with a state of the art of perception 
algorithms and techniques grouped by competences. Sensors used on each work
are enumerated, and their advantages and disadvantages discussed. 
Section \ref{sec:04-relevantdemos} gives a perspective of the evolution of
perception in Automated Driving, presenting the most relevant works and demos
in the history of the discipline with a focus in sensor technologies used for
each one. 
Finally, section \ref{sec:06-discussion} contains a discussion of the current
state of the discipline and the future challenges for sensors and perception in
Automated Driving systems. It includes a review of the most relevant alliances 
between OEMs and technological companies involved in automated driving projects
at the time of writing the article.
