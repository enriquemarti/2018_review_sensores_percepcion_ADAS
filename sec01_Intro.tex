%Hacer referencia a los accidentes, o plantear otra motivación?
Every year more than one milion people die on road accidents, and several 
million more get injured. In addition to the social cost, it also has an
important economic impact for nations worldwide. According to 
\cite{Thomas2013}, the most frequent causes for car accidents in the
European Union are human related: speeding, driving under the effects of
alcohol or drugs, reckless driving, distractions or just plain misjudgments.
This is likely to be extrapolable to the rest of the world.
Automated Driving systems, also called self-driving vehicles, aim to take the 
human driver out of the equation. Thus, they are thought to be a valuable tool
towards reducing the number of traffic accidents.
%Falta completar con datos reales y referencias.

Based on recent developments and demonstrations around the world, there is a 
tendency to think that Automated Driving with a high level of automation will 
be available in a few years. 
Highly advanced ADAS, like Adaptative Cruise Control (ACC), Automatic Emergency
Braking(AEB) or Lane Keep Asistant (LKA) are currently avalible at the market 
and highly accepted between users. Also, some recent demos like Waymo or Tesla 
(discused in detail in Section \ref{sec:04-relevantdemos}) shows how future 
autonomous vehicles will be.
However, there are still many research challenges, such as navigation in urban 
dynamic environments, accurate obstacle avoidance capabilities, environment 
understanding in real-time, and perception uncertainties among others. This 
research is needed to allow cooperative maneuvers between automated and 
semiautomated vehicles, which still need further efforts in real 
implementation, specifically in urban environment.

The architecture of autonomous vehicles is usually divided into three 
categories: Perception of the environment, behaviour planning and motion 
execution \ref{XXXX}[Sagar Behere](sense-plan-act). The first category, 
perception of the environment, can be challenging for complex and dynamic 
environments. Autonomous vehicles obtain data of their sorrounding making use 
of different sensors, such as cameras, LiDARS and radars. This raw data is then 
processed to extract relevant characteristics which are the input of the 
following categories (behaviour planning and motion execution), that will 
perform tasks such as path planning, collision avoidance or control of the 
vehicle among others \ref{XXXX}[Bagloee2016].

This paper pays special attention to the sensor and perception stages.
They especially important because they are the first link in the chain of
Automated Driving, and a failure in that stage is propagated and has a chance
of producing a severe accident. Some examples in the real world include
the 2016 Tesla AutoPilot accident \cite{NTSB2017}, where a man was killed 
after its car crashed a truck: the camera failed to detect the truck because
it was painted in a color similar to the bright sky, while at the same time 
the radar detection was descarded as background noise by perception algorithms.
Later in 2018, other Tesla vehicles have crashed against highway dividers
after the lane following system failed to detect faded lines.
Also in 2018, an experimental self-driving Uber vehicle killed a woman
crossing the road \cite{NTSB2018} in the night, dressed in dark clothes. 
Only the LiDAR provided a positive detection, that was discarded as a false
positive by perception algorithms.

Automated vehicles need to have a good understanding of the surrounding 
environment in order to drive safely in the roads. A state of the art of the 
different configurations and sensors used by different manufacturers, based on 
the most relevant demonstrators on Automated Driving, developed by research 
institutions and manufactures is presented. This topic has been widely 
investigated in the literature, but based usually on ADAS implementation. It is 
a key aspect in the future developments highly automated cars, where Real Time 
Motion planning needs more accurate and robust inputs. This paper makes an 
overview and analysis of the main problems, applications and sensor 
technologies available in the market.

The content of this paper is organized as follows. Section \ref{sec:02-sensors} 
describes the 
sensors commonly used for perception explaining the technologies that they use 
and the new emerging ones that will be used in the future. The most important 
competences in perception are discused in section
\ref{sec:03-problemsapplications}, describing the different sensors that can
used in each competence and discussing their advantages and disadvantages. 
Section \ref{sec:04-relevantdemos} presents the most relevant works and demos, 
describing the kind of sensors used for each one. Finally, section
\ref{sec:06-discussion} presents the most relevant alliances between OEMs and 
technological companies involved in automated driving projects, and contains a
final discussion of the future challenges for sensors and perception in
Automated Driving systems.
